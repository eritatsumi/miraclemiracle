%\documentclass[review]{elsarticle}
%\documentclass[preprint,11pt,authoryear]{elsarticle}
%\documentclass[3p,twocolumn,authoryear]{elsarticle}
\documentclass[3p,authoryear]{elsarticle}
%----tatsumi
\usepackage{lscape}
\usepackage{listings}
\usepackage{comment}
%\usepackage{doublespace}
\usepackage{amsmath,amssymb}
%\usepackage{ascmac}
%\usepackage{bm}
%\usepackage[dvipdfmx]{graphicx}
\usepackage{longtable}
\usepackage{mediabb}
\usepackage{color}
\usepackage{ulem}
%\usepackage{setspace}
%\usepackage{showkeys}
\newcommand{\ma}[2]{\textcolor{red}{#1 \sout{#2}} }
\newcommand{\red}[1]{\textcolor{red}{#1}}
\newcommand{\gr}[1]{\textcolor{green}{#1}}
%\newcommand{\ma}[2]{#1}

\newcommand{\del}{\partial}
\renewcommand{\d}{{\rm{d}}}
\newcommand{\II}{I\hspace{-.1 em}I}
\newcommand{\III}{I\hspace{-.1 em}I\hspace{-.1 em}I}

%--------

\usepackage[dvipdfmx]{hyperref}
\hypersetup{colorlinks=true}
\usepackage{lineno}
\modulolinenumbers[5]

\journal{Icarus}

%%%%%%%%%%%%%%%%%%%%%%%
%% Elsevier bibliography styles
%%%%%%%%%%%%%%%%%%%%%%%
%% To change the style, put a % in front of the second line of the current style and
%% remove the % from the second line of the style you would like to use.
%%%%%%%%%%%%%%%%%%%%%%%

%% Numbered
%\bibliographystyle{model1-num-names}

%% Numbered without titles
%\bibliographystyle{model1a-num-names}

%% Harvard
%\bibliographystyle{model2-names.bst}\biboptions{authoryear}

%% Vancouver numbered
%\usepackage{numcompress}\bibliographystyle{model3-num-names}

%% Vancouver name/year
%\usepackage{numcompress}\bibliographystyle{model4-names}\biboptions{authoryear}

%% APA style
%\bibliographystyle{model5-names}\biboptions{authoryear}

%% AMA style
%\usepackage{numcompress}\bibliographystyle{model6-num-names}

%% `Elsevier LaTeX' style
%\bibliographystyle{elsarticle-num}
\bibliographystyle{elsarticle-harv}
%%%%%%%%%%%%%%%%%%%%%%%

\begin{document}

\begin{frontmatter}

\title{Cratering efficiency on coarse-grain targets: implications for the dynamical evolution of asteroid 25143 Itokawa}
%\tnotetext[mytitlenote]{Fully documented templates are available in the elsarticle package on \href{http://www.ctan.org/tex-archive/macros/latex/contrib/elsarticle}{CTAN}.}

%% Group authors per affiliation:
\author[affiliation1]{Eri Tatsumi\corref{mycorrespondingauthor} }
\author[affiliation1,affiliation2]{Seiji Sugita}
\address[affiliation1]{Department of Earth and Planetary Science, the University of Tokyo, 7-3-1 Hongo, Bunkyo-ku, Tokyo 113-0033 Japan}
\address[affiliation2]{Research Center for the Early Universe, the University of Tokyo, 7-3-1 Hongo, Bunkyo-ku, Tokyo 113-0033 Japan}
%\fntext[myfootnote]{Since 1880.}

%% or include affiliations in footnotes:
\author{}
\cortext[mycorrespondingauthor]{Corresponding author}
\ead{eri@eps.s.u-tokyo.ac.jp}
%\ead[url]{www.elsevier.com}


\begin{abstract}
Remote sensing observations made by the spacecraft Hayabusa provided first direct evidence for a rubble-pile asteroid: 25143 Itokawa.
Itokawa was found to have a surface structure very different from other explored asteroids; covered with coarse pebbles and boulders
ranging at least from cm to meter size. The cumulative size distribution of small circular depressions on Itokawa, most of which may be
of impact origin, has a significantly shallower slope than that on the Moon; small craters are highly depleted on Itokawa compared to the Moon.
The deficiency in small circular depressions and other features, such as clustered fragments and pits on boulders, suggest boulders on
Itokawa might behave as armor and prevent crater formation: ``armoring effect''.
\red{This might be one factor that causes the depletion in the number of small craters.}
In this study, the cratering efficiency reduction due to coarse-grained targets is investigated based on impact experiments at velocities
ranging from $\sim 70$ m/s to $\sim 6$ km/s using two vertical gas gun ranges.
We then propose a scaling rule extended for cratering on coarse-grained targets (i.e., target grain size $\gtrsim$ projectile size).
We have found that the crater efficiency reduction is caused by energy dissipation at the collision site where momentum is transferred
from the impactor to the first-contact target grain, and that the armoring effect can be classified into three regimes, (1) gravity scaled regime,
(2) reduced size crater regime, and (3) no apparent crater regime, depending on the ratio of the impactor size to the target grain size
and the ratio of the impactor kinetic energy to the disruption energy of a target grain.
\red{We found that the shallow slope of the circular depressions on Itokawa can not be account for by this new scaling law, and that obliteration processes, such as regolith convection and migration, play a great role for the depletion of circular depressions on Itokawa.}
Based on the new extended scaling law, we found that the crater retention age on Itokawa is 10 -- 33 Myr in the main belt, in good agreement
with in the cosmic-ray exposure ages for returned samples from Itokawa which may reflect the age of a few meter beneath the surface.
These ages strongly suggest that the global resurface resetting the 1 -- 10 m deep surface layer may have occurred in the main belt long
after a possible catastrophic disruption of a rigid parent body of Itokawa suggested by Ar degassing age ($\sim 1.3$ Gyr).
\end{abstract}

\begin{keyword}
Cratering\sep Impact process\sep Asteroid surfaces \sep Asteroid Itokawa \sep Asteroids
\end{keyword}

\end{frontmatter}

\tableofcontents
\linenumbers
%%%%%%%%%%%%%%%%%%%%%%%%%%%%%%%%%%%%%%%%%%%%%%%%%%%%%%%%%%%%%%%%%%%%%%%%%%%%%%%%%%%%%%%%%%%%%%%%%%%

\section{Introduction}\label{sec:intro}
%%%%%%%%%%%%%%%%%%%%%%%%%%%%%%%%%%%%%%%%%%%%%%%%%%%%%%%%%%%%%%%%%%%%%%%%%%%%%%%%%%%%%%%%%%%%%%%%%%%

Japanese spacecraft Hayabusa made remote sensing observations of the near-Earth asteroid 25143 Itokawa on 2005 \citep{fujiwara2006, saito2006}. 
This was the first spacecraft observation of a rubble-pile asteroid and provided us with a variety of information on surface structures of a rubble-pile asteroid.

``Rubble-pile'' objects,  loosely bound and gravity-dominated aggregates with negligible tensile strength \citep[e.g.][]{fujiwara1980}, are now
widely accepted to be common among small asteroids \citep[e.g.][]{richardson2002}. The bi-modal distribution of rotational periods strongly suggest
that there is a threshold of rotation period and many asteroids $0.15<D_a<10$ km are rubble piles \citep{pravec2002}. More recently, both rock
friction calculations suggest that asteroids smaller than 0.15 km in diameter are also loosely combined with small cohesion against
centrifugal force beyond the spin barrier \citep[e.g.][]{holsapple2007,scheeres2010,rozitis2014}. Moreover, bulk density measurements suggests that the porosities of asteroids smaller
than a few tens km are large \citep{britt2002}. The total mass of the main belt asteroids (MBAs) was estimated as $3.6\times 10^{21}$ kg from an
analysis on the motion of Mars, constraining the contribution of asteroids too small to be observed individually with telescopes \citep{krasinsky2002}.
Subtracting the known masses of the largest asteroids \citep{demeo2014} from the total mass of the main belt, the minor asteroids ($<$20 km) would account for $\sim 35 \%$ of the total mass of MBAs. Another characteristics of small asteroids that can be observed by ground-based telescopes is thermal
inertia, which allows us to estimate the surface physical conditions, such as grain size and inter-grain contact state. Typical km-sized asteroids
have thermal inertia higher than large asteroids (200 -- 1000 km) by more than one order of magnitude, suggesting the small asteroids are
expected to have less regolith and more coarse-grained surface \citep{delbo2009}. For example, the thermal inertia of Itokawa is $\sim 750$ Jm$^2$s$^{0.5}$K$^{-1}$\citep{muller2005}, while that of the Moon is $\sim 40$ Jm$^2$s$^{0.5}$K$^{-1}$\citep{Keihm1984}.
This observation is consistent with the short collisional lifetimes of small asteroids. 
Such short surface age would prevent surface grains from grinding down to fine grains due to impact and/or thermal fatigue fragmentation.
Also, small gravity and escape velocities of these small bodies may allow electrostatic forces to cause micro size dust grains to leave the asteroids rather efficiently \citep{lee1996,hartzell2011,nagao2011}.
Thus, the small members of asteroids are possibly dominated by coarse-grained surfaces.
Moreover, two sample return mission target asteroids, Ryugu by Japanese Aerospace Exploration Agency (JAXA) and Bennu by National Aeronautics
and Space Administration (NASA), have slightly smaller thermal inertias \citep{emery2014, hasegawa2008} than Itokawa but significantly larger than
that of fine regolith, suggesting coarse-grained surface as well.

The asteroid Itokawa may be a good representative for such rubble-pile asteroids with coarse-grained surfaces.
The surface of Itokawa has two distinct types of terrains, rough terrains, which occupies $\sim 80\%$ of its surface, and smooth terrains \citep{saito2006}.
Most of the surface is covered with cm to m-sized angular pebbles and boulders. The size range of boulders is rather large compared to Itokawa's small body (0.54 km $\times$ 0.29 km $\times$ 0.21 km).

Another type of distinctive geological features on Itokawa are circular depressions, which are formed likely by meteoroidal impacts \citep{hirata2009}.
The number density of large circular depressions ($\gtrsim 100$ m) are comparable to that of the Moon.
However, the cumulative size distribution of small circular depressions on Itokawa has a significantly shallower slope than that on the Moon;
small craters are highly depleted on Itokawa compared to the Moon.
The reason for the depletion in small depressions could be caused by the combinations of production rate reduction and erasure processes.
Smaller ones could be erased faster than larger ones which may be preserved longer time. Regolith migration \citep{miyamoto2007, tancredi2015}
might deform and obliterate these features.
On the other hand, close-up images of Itokawa surface with pixel resolution of $\sim 6$ mm to $\sim70$ cm taken by the Asteroid Multi-band Imaging
CAmera (AMICA) provide imprints for small impact processes other than excavated crater-like features (Figure \ref{amica}). \citet{nakamura2008} found
several pieces of evidence in these AMICA images for the presence of small meteorite collisions to boulders on the surface of Itokawa.
For example, possible boulder fragments (Figure \ref{amica}(a)) with angular and conical shape and numerous bright spots  on some boulders (the arrows in Figures \ref{amica}(b) and (c)) which probably indicate fresh spalled areas by small impacts.
Those features suggest the presence of meteoroidal collisions without excavating subsurface materials because the disruption energy of surface large boulders is large enough to absorb a large fraction of impact energy.

Especially on small km-sized asteroids, whose surfaces are covered with coarse grains, the condition that the target grain size is comparable or larger than a projectile size occurs frequently. Such impacts on large grains may cause so-called ``armoring effect'', in which large boulders on asteroid surface behave as armor.
This effect was predicted in previous studies \citep[e.g.,][]{barnouin2005, guettler2012}.
However, quantitative analyses for armoring effect under realistic conditions, i.e., cratering involving extensive disruption of target grains, has not been obtained; several experiments under limited condition were performed.
Thus, there is large uncertainty in cratering scaling on coarse-grained (target-grain size/projectile size $\gtrsim $ unity) surfaces. Due to this uncertainty accurate age estimate cannot be made based on crater counting on Itokawa surfaces. An accurate age estimate for rubble-pile asteroids needs an accurate crater scaling law.

The goal of this study is to evaluate the cratering efficiency on coarse-grained targets and investigate the armoring mechanism.
To achieve this goal, first we review previous experimental studies (Section \ref{sec:review}). Then, we performed impact experiments using coarse-grained targets that simulate rubble-pile asteroids surfaces (Section \ref{sec:experiment}).
More specifically, impact crater dimensions are measured and analyzed based on dimensional analysis.
First, we analyzed based on the classic $\pi$ scaling derived by \citet{holsapple1993} (Section \ref{sec:classic-pi}).
Then, we formulate an extended $\pi$ scaling law to estimate crater size on coarse-grained targets in order to apply for asteroid surfaces (Section \ref{sec:extension-pi}) and discuss the armoring mechanism through observations (Section \ref{sec:mechanism}).
Furthermore, planetary implications including the crater retention age of Itokawa and its dynamical history, and the mass evolution of the main asteroid belt are discussed (Section \ref{sec:implication}) before conclusions (Section \ref{sec:conclusion}).

\begin{figure*}[tbp]
	\begin{center}
	\includegraphics[width=\textwidth]{/Users/eri/Dropbox/2016icarus/Figures/fig1.eps}
	\caption{Itokawa's collisional imprints. (a) Angular and conical boulders' cluster (dashed enclosure) and a write spot (arrow) (ST{\_}2530286817). (b) White spots (arrows) are considered to be relatively fresh areas (ST{\_}2539444467). (c) A zoom up image of white spots (ST{\_}2566271576). (d) A crater-like circular depression whose floor is filled with fine particles. \citep{nakamura2008,hirata2009}}
	\label{amica}
	\end{center}
\end{figure*}

%------------------------------------------------------------%
\section{Previous experiments on coarse-grained targets} \label{sec:review}
%------------------------------------------------------------%
Several experiments on coarse-grained targets have been conducted and revealed that the ratio of target grain size to projectile size influences the cratering process greatly \citep{cintala1999, barnouin2005, guettler2012, holsapple2014}.
These experiments, however, are not necessarily consistent with each other, especially with respect to velocity dependence.
The first series of experiments on this issue was conducted by \citet{cintala1999}. They used the 1 -- 3 mm coarse-grained sand as targets and aluminum spheres of 4.76 mm in diameter as impactors. In their experiments impactors and target grains are comparable in size, although impactors were still larger than grain size.
Although their primary interest was ejecta velocity, they have also found that crater size is the same as crater size on fine-grained sand and that crater shapes with the depth/diameter ratios of $\sim 0.23$ are similar to simple craters.
More recently, Hayabusa mission revealed that the asteroid Itokawa possesses a surface condition very different from other explored asteroids \citep[e.g.][]{fujiwara2006,saito2006}, attracting much attention to cratering processes on coarse-grained targets \citep{barnouin2011, guettler2012, holsapple2014}.
\citet{barnouin2011} reported different behaviors between fine-grained targets and coarse-grained targets at low impact velocities $<1.5$ km/s.
More systematic experiments at low speeds ($\sim$200 -- 300 m/s) on coarse-grained glass sphere targets with a variety of the projectile/ target grain size by \citet{guettler2012} revealed that  the crater sizes on the coarse-grained targets are controlled by the projectile/ target grain size.
More specifically, \citet{guettler2012} found that for a given the projectile/ target grain size, the armoring effect of crater reduction from the gravity scaling do not depend on impact velocity within their velocity range.
However, this estimation may be applicable to low-velocity cratering. Impact experiments by \citet{barnouin2011} suggest that armoring effect may depend on impact velocity.
More recent impact experiments by \citet{holsapple2014} at velocities high enough to fully disrupt a target grain revealed that craters formed under such conditions are as large as predicted with a  classical gravity scaling rule (i.e., dry sand cases), suggesting that armoring effect observed at low velocities may not be very pronounced at high velocities $\sim 5$ km/s.

In order to understand this situation clearly, we use the ratio  $\xi =\frac{1}{2}m_pU^2/m_tQ_D^*$ of the impact energy to the disruption energy of a target grain as well as the size ratio $\psi=D_p/D_t$ of a projectile to target grains and analyzed the relationship between the degree of armoring effect and experimental conditions for previous experiments. We plotted the degree of armoring effect in the $\xi - \psi$ space (Figure \ref{phase}). Figure \ref{phase} clearly shows that the degree of armoring effect changes with both size ratio $\psi$ and energy ratio $\xi$.
More specifically, the experimental result of \citet{holsapple2014} exhibits that the conclusion of \citet{guettler2012} that the armoring effect occurs when $\psi<1$ is not always correct and impact energy might influence the degree of armoring.
%Although the energy ratio was not considered in the previous studies, we found that the energy ratio might have large influence on the degree of armoring.
% The previous experimental conditions of coarse-grained targets are depicted in Figure \ref{phase} using these parameters.
This suggests that the size ratio is not the only controlling parameter for the armoring condition and that the energy ratio should be taken into account.
The condition in which crater size was reduced is shown in the dashed circle in Figure \ref{phase}. Otherwise, at least under the previous experimental condition, the crater size was comparable to what the classic gravity scaling by \citet{holsapple1993} predicts.
We evaluate the transition from the armoring to the classic gravity scaling which may be found between these conditions.
\begin{figure}[htbp]
\begin{center}
	\includegraphics[width=80mm]{/Users/eri/Dropbox/2016icarus/Figures/fig2_new.pdf}
	\caption{A phase diagram for armoring effect; which styles of cratering occur is shown as a function of $\psi$ (projectile size/ target grain size) and $\xi$ (impact energy/ catastrophic disruption energy for one grain) from previous experiments \citep{cintala1999,guettler2012,holsapple2014}. Open symbols indicate gravity-scaled-sized craters \citep{holsapple1993} and filled symbols indicate craters smaller than what the gravity scaling predicts. Here, experimental results that fall within a factor of two (i.e., $+100\%$ and $-50\%$) from the gravity scaling are shown with open symbols and data points below this range are shown with filled symbols.}
	\label{phase}
\end{center}
\end{figure}

 %%%%%%%%%%%%%%%%%%%%%%%%%%%%%%%%%%%%%%%%%%%%%%%%%%%%%%%%%%%%%%%%%%%%%%%%%%
 \section{Impact experiments} \label{sec:experiment}
 %%%%%%%%%%%%%%%%%%%%%%%%%%%%%%%%%%%%%%%%%%%%%%%%%%%%%%%%%%%%%%%%%%%%%%%%%%
Our interest in this study is to understand cratering on rubble-pile asteroids with fragmentation of grains, where projectiles are comparable to or smaller than target grains in size and impact energy is larger than the disruption energy of one grain.
Previous studies did not compare experimental results with the ratio $\xi$ of impact energy to disruption energy of a target grain, but they only take into account the size ratio $\psi$.
Here, we add the energy ratio $\xi$ parameter and conduct experiments under various combinations of the energy ratio $\xi$ and the size ratio $\psi$.

%------------------------------------------------------------%
 %\subsection{Experimental conditions}
%------------------------------------------------------------%

We used two vertical guns in the University of Tokyo and Institute of Space and Astronautical Science (ISAS/JAXA) , for impact velocities 79 -- 224 m/s and 1 -- 6 km/s, respectively (Figure \ref{gun_range}). We used polycarbonate projectiles 0.76 -- 0.77 g  in mass and 10 mm in diameter for the former gun and 0.068 g in mass and 4.7 mm in diameter for the latter gun, respectively.
The shapes of the projectiles are ``bullet shapes'' with spherical head and cylindrical tail and spheres for the former gun and for the latter gun, respectively.

We used pumice with different mean grain diameters ($\sim$7, 9 and 16 mm), basalt ($\sim$ 10 and 18 mm), sintered-glass-beads (SGB) ($\sim 12$ mm) and soda glass spheres ($10$ mm) as boulder target simulants.
By using weakly sintered-glass-beads \citep[e.g.][]{setoh2010} as target materials, we reproduced the conditions of cratering involving disruption at relatively low impact velocities achievable with our facilities.
 The material properties of target materials used in this study are summarized in Table \ref{material}, and the grain size distributions are given in Figure \ref{grainsize-CDF}, where all the averages of mean diameters (average of long axis and short axis) have comparable to or greater than impactor size (i.e., $\psi \gtrsim 1$).
Because the specific energy for catastrophic disruption of basalt, pumice and sintered-glass-beads are $\sim 1000$ J/kg, $\sim 2000$ J/kg and $\sim 30$ J/kg \citep{takagi1984,HH1999,setoh2010,patmore2014,flynn2015}, energies required to disrupt a 1-cm grain are $\sim 9$ J, $\sim 7$ J and $\sim 0.2$ J, respectively.
A sintered-glass-beads grain is easily disrupted at impact velocities around 100 m/s.
The schematic configuration of experiments is illustrated in Figure \ref{schematic-exp}. Impact velocity was measured by a pair of laser detecting systems.
 Note that all the impacts were vertical, and each shot was recorded with high-speed cameras (NAC, Fx-4 and Q1v) at 5000 -- 8000 fps of framing rates.
 Some of the experiments were performed with quarter-space targets to observe the impact process underneath the target surface.
We measured the final rim-to-rim diameter and depth of craters. The cross-section crater profiles of high-velocity cases were obtained with a high-precision laser profiler (Keyence, LJ-V).
However, the shapes of the craters are sometimes difficult to determine, especially for low-velocity cases, because crater sizes become comparable to large target grains.
In order to resolve this problem, we colored the pre-impact targets surface with paint spray to make it noticeable which grains have moved upon impact.
This technique turned out to be very effective and allowed us to discern which portion of the target is excavated (Figure \ref{crater-splay}).

\begin{figure*}[tbhp]
	\centering
	\includegraphics[width=120mm]{/Users/eri/Dropbox/2016icarus/Figures/fig3.eps}
	\caption{Two vertical gas gun ranges used in this study. (a) The two-stage light gas gun range in ISAS/JAXA with velocity range of $\sim 1\,-\,6$ km/s. (b) The one-stage light gas gun range in the University of Tokyo with velocity range of $\sim $ 79 -- 224 m/s.}
	\label{gun_range}
	\centering
\end{figure*}
\begin{figure}[htbp]
	\begin{center}
	\includegraphics[height=70mm]{/Users/eri/Dropbox/2016icarus/Figures/fig4.eps}
	\caption{A schematic diagram for our experimental setting at the Univ. of Tokyo. }
	\label{schematic-exp}
	\end{center}
\end{figure}
\begin{table*}
	\caption{The mechanical properties of the materials used as targets.}
	\label{material}
	\begin{center}
	\small
	\begin{tabular}{lcccc}\hline
	Material & Density (g/cm$^3$) & Comp. Str. (MPa) & $Q_D^*$ (J/kg) & $Q_V^*$ (J/m$^3$) \\ \hline
	Sintered-glass-beads & $1.48\pm0.15$ & 0.5 - 5$^{\rm (a)}$ & 34 & $\sim 5.1\times 10^4$\\
	Pumice & $0.72 \pm 0.1$ & $854\pm 195^{\rm (b)}$& 2380$^{\rm (c)}$& $\sim 1.7\times 10^6$\\
	Basalt &$2.7\pm0.1$ & $\sim 100-300^{\rm (d)}$&$\sim 800 -1500^{\rm (d)}$& $\sim 2.2\times 10^6$\\
	 \hline
	\end{tabular}\\
	(a) \citet{setoh2010}, (b) \citet{patmore2014}, (c) \citet{flynn2015}, (d) \citet{takagi1984,HH1999}.
	\end{center}
\end{table*}
%
\begin{table*}[btp]
	\caption{The size and mass of the target grains.}
	\centering
	\small
	\begin{tabular}{lcccc} \hline
	Target name &Target material & Median mass & Average mass& Average mean diameter\\
	& &(g) &$m_t$ (g)& $D_t$ (mm) \\ \hline
	Basalt-m & basalt & 0.83 & $1.06\pm 0.63$ & 9.9\\
	Basalt-l & basalt & 6.25 & $6.46 \pm 2.50$ & 18.1\\
	Pumice-s& pumice & 0.094 & $0.106\pm0.045$ & 7.0\\
	Pumice-m & pumice & 0.220 & $0.232\pm0.077$ &9.1 \\
	Pumice-l & pumice & 1.20 & $1.29\pm 0.57$ & 15.6\\
	Sintered-GB & Sintered-glass-beads & 1.00 & $1.05 \pm 0.45$ & 11.7\\
	\hline
	\end{tabular}
	\centering
\end{table*}
\begin{figure*}[tbp]
	\begin{center}
	\includegraphics[width=\textwidth]{/Users/eri/Dropbox/2016icarus/Figures/fig5.eps}
	\caption{Size distributions of target grains. Cumulative frequency distribution of volumes of target grains (left) and masses and mean diameters of target grains (right). Approximately 90 grains were randomly chosen from each target.}
	\label{grainsize-CDF}
	\end{center}
\end{figure*}


\begin{figure}[htbp]
	\begin{center}
	\includegraphics[width=80mm]{/Users/eri/Dropbox/2016icarus/Figures/fig6.eps}
	\caption{An example of pumice target with color-sprayed on the surface (a) before and (b) after a cratering experiment.}
	\label{crater-splay}
	\end{center}
\end{figure}

%------------------------------------------------------------%
 \section{Results and analyses}\label{sec:classic-pi}
%------------------------------------------------------------%
The impact conditions and resulting crater dimensions of all the cratering experiments in this study are given in Table \ref{exp-summary}. Figures \ref{crater-low} and \ref{crater-high} show the final craters and their radial profiles of topographies we obtained in the experiments. There is a rather drastic change in crater morphology and size as a function of impact conditions. More specifically, for low impact energies, crater shapes deviate greatly from simple craters; irregular shapes and shallow craters occur (Figures \ref{crater-low} (a), (b), and (c)). This observation may be explained if the excavation flow dose not develop fully because the low energy is dissipated mostly by the disruption of the first-contact grain. For high energy impacts, the shapes are very similar to simple craters (Figures \ref{crater-low} (d) and \ref{crater-high}). In order to evaluate such change more quantitatively, we compared our experimental data with classical $\pi$ scalings: gravity scaling and strength scaling \citep[e.g.][]{holsapple1993}.
 %
% \begin{landscape}
 {\footnotesize
%\begin{table}[htbp]
%	\begin{center}
%	\begin{tabular}{ll|cccc}
	\begin{longtable}[c]{ll|cccc}
	\caption{Experimental conditions and outcomes. \label{exp-summary}}\\
	\hline
	Exp. No. & Target name&Projectile mass& Impact velocity & Crater diameter &Crater depth \\
	& &$m_p$& $U$ & $D_c$ &  $H$ \\
	& & (g)& (m/s) & (mm)& (mm) \\ \hline
	\endfirsthead
	\multicolumn{6}{l}{{ \it continued from previous page}}\\ \hline
	Exp. No. & Target name &Projectile mass& Impact velocity & Crater diameter &Crater depth \\
	& &$m_p$& $U$ & $D_c$ &  $H$ \\
	& & (g)& (m/s) & (mm)& (mm) \\ \hline
	\endhead
	\hline \multicolumn{6}{r}{{ \it Continue on next page}} \\ \hline
	\endfoot
	\endlastfoot
	SGB001& Sintered-GB & 0.77 & 108 & $89\pm 5$ & $20\pm 5$\\
	SGB002& Sintered-GB & 0.77 & 112 & $89 \pm 6$ & $23\pm5$\\
	SGB003& Sintered-GB & 0.77 & 127 & $98 \pm 8$ & $26\pm 5$\\
	SGB004& Sintered-GB & 0.77 & 136 & $97\pm 5$ & $24\pm5$\\
	SGB005& Sintered-GB & 0.77 & 152 & $95 \pm12$ & $29\pm5$\\
	SGB006& Sintered-GB & 0.77 & 162 & $103\pm 5$ & $28\pm5$ \\
	SGB007& Sintered-GB & 0.77 & 170 & $100 \pm 5$ & $25 \pm5$\\
	SGB008& Sintered-GB & 0.77 & 177 & $112\pm12$ & $33 \pm5$ \\
	SGB009& Sintered-GB & 0.77 & 194 & $103 \pm 6$ & $33\pm 5$ \\
	SGB010& Sintered-GB & 0.77 & 224 & $107 \pm 8$ & $28 \pm 5$\\ \hline
	P001&Pumice-m & 0.76 & 79 & $94 \pm 11$ &$18 \pm 5 $\\
	P002&Pumice-m & 0.76 & 84 & $100 \pm15$ &$16 \pm 5 $\\
	P003&Pumice-m & 0.76 & 85 & $105 \pm12 $&$16 \pm 5 $\\
	P004&Pumice-m & 0.76 & 92 & $106 \pm 11$ & $18 \pm5 $\\
	P005& Pumice-m & 0.76 & 105 & $95 \pm7$ & $26 \pm 5$ \\
	P006& Pumice-m & 0.76 & 109 &  $92 \pm 13$ & $24 \pm 5$ \\
	P007 & Pumice-m & 0.76 & 110 & $101 \pm 14$ & $20 \pm 5$\\
	P008 & Pumice-m & 0.76 & 130 & $108 \pm 12$ & $17 \pm5$\\
	P009& Pumice-m &0.76 & 132 & $118 \pm 7 $& $19 \pm 5$\\
	P010& Pumice-m & 0.76 & 144  & $106 \pm 10$ & $19 \pm 5$\\
	P011& Pumice-m & 0.76 & 144 & $125 \pm 14$ & $16 \pm 5$\\
	P012& Pumice-m &0.76 & 155 & $126 \pm 32$ & $18 \pm 5 $\\
	P013& Pumice-m & 0.76 & 156& $119 \pm 12$ & $21 \pm 5$\\
	P014& Pumice-m & 0.76 & 169 & $96 \pm 15$ & $19 \pm 5$ \\
	P015& Pumice-m & 0.76 & 173& $120 \pm 21$ & $20 \pm 5$ \\
	P016& Pumice-m &0.76 & 175  & $118 \pm 11$ & $18 \pm5$ \\
	P017 & Pumice-m & 0.76 & 193 & $126\pm15$ & $25\pm5 $ \\
	P101& Pumice-m & 0.76 & 129 & $102 \pm 5$ & $24\pm 5$\\
	P102& Pumice-m &0.76 & 140 & $99 \pm 5 $& $25 \pm 5$\\
	P103& Pumice-m &0.76 & 184 & $114 \pm 5$ & $29 \pm 5$\\
	P104& Pumice-m &0.76 & 188 & $112 \pm 5$ & $27 \pm 5$\\
	P105& Pumice-m &0.76 & 201 & $121 \pm 5$ & $27 \pm 5$\\
	P106& Pumice-m & 0.76 & 203 & $101 \pm 5$ & $25\pm5$\\
	P107& Pumice-m &0.76 & 219 & $107 \pm 5 $& $40 \pm 5$ \\
	P201  & Pumice-m & 0.068 & $0.98 \times 10^3$ & $96\pm 11$ & $18\pm 5 $ \\
	P202 &Pumice-m& 0.068 &$1.57\times 10^3$ & $127\pm14$ & $24\pm5$\\
	P203 & Pumice-m & 0.068 &$2.06 \times 10^3 $ & $161\pm14 $ & $34\pm 5$\\
	P204 & Pumice-m & 0.068 & $2.83\times10^3 $ & $193 \pm 17$ & $32 \pm 5$\\
	P205 & Pumice-m & 0.068 & $3.60 \times 10^3 $ & $215\pm 13 $ & $ 44 \pm 5 $ \\
	P206*&Pumice-m & 0.068 & $4.33\times 10^3$ & $240\pm10$ & $48\pm5$\\
	P207 & Pumice-m & 0.068 & $5.05 \times 10^3$ & $280 \pm 13$ & $56\pm 5$ \\
	P208*& Pumice-m & 0.068 & $5.49\times 10^3$ & $310\pm10$ & $58\pm5$\\
	P209 & Pumice-m & 0.068 &$6.05\times 10^3$ & $277\pm 16$ & $51\pm 5$ \\
	\hline
	P301& Pumice-s & 0.74 & 122 & $126\pm6$ & $17\pm5$\\
	P302& Pumice-s & 0.74 & 132 & $139\pm7$ & $16\pm5$\\
	P303& Pumice-s & 0.74 & 144 & $118\pm5$ & $17\pm5$\\
	P304& Pumice-s & 0.74 & 152 & $124\pm5$ & $19\pm5$ \\
	P305& Pumice-s & 0.74 & 163 & $121\pm5$ & $23\pm5$\\
	P306& Pumice-s & 0.74 & 165 & $131\pm5$ &  $19\pm5$\\
	P307& Pumice-s & 0.74 & 184 & $118\pm16$ & $22\pm5$\\
	P308& Pumice-s & 0.74 & 186 & $131\pm8$ & $18\pm5$\\
	P401&Pumice-s & 0.068 & $0.99\times 10^3$ & $99 \pm10$ & $17\pm5$\\
	P402& Pumice-s&0.068 & $1.70\times 10^3 $&$147\pm14$ & $32\pm 5$\\
	P403& Pumice-s&0.068 & $3.32\times 10^3$& $200\pm28$ & $42\pm5$\\
	P404& Pumice-s & 0.068 & $4.59\times 10^3$&$250\pm19$ & $49\pm 5$\\
	P405& Pumice-s & 0.068 & $5.92\times 10^3$ & $276\pm11$ & $64\pm5$\\
	\hline
	P501& Pumice-l & 0.74 & 183 & $92\pm 12$ &$23 \pm 5$\\
	P502& Pumice-l & 0.74 & 69 & $32\pm 7$ & $8\pm5$\\
	P503& Pumice-l & 0.74 & 115 & $55\pm 5$ & $14\pm 5$\\
	P504& Pumice-l & 0.74 & 145 & $60\pm20$ & $15\pm5$\\
	P505& Pumice-l & 0.74 & 160 & $78\pm12$ & $20\pm5$\\
	P601*&Pumice-l& 0.068 & $0.96\times 10^3$ &$110\pm10$& $20\pm5$\\
	P602&Pumice-l&0.068 & $1.78\times 10^3$ & $130\pm18$ & $22\pm5$\\
	P603&Pumice-l & 0.068 & $3.15\times 10^3$ & $163\pm 16$ &$35\pm5$\\
	P604& Pumice-l & 0.068 & $4.43\times 10^3$ & $206\pm 23$ & $44\pm5$\\
	P605*& Pumice-l & 0.068 & $5.34\times 10^3$ & $220\pm 10$ & $53\pm5$\\
	P606& Pumice-l & 0.068 & $5.95\times 10^3$ & $275\pm19$ & $66\pm5$\\
	\hline
	B001*&Basalt-m& 0.068 & $0.88\times 10^3$ & $108\pm8$ & $12.5\pm5$\\
	B002*&Basalt-m & 0.068 & $2.53\times 10^3$ & $135\pm 5$ & $38\pm5$\\
	B003*&Basalt-m & 0.068 & $4.30\times 10^3$ & $240\pm10 $&$53\pm5$\\
	B004*&Basalt-m & 0.068 & $5.91\times 10^3$ & $205\pm 5$ & $48\pm5$\\
	B005&Basalt-m & 0.068 &$6.10\times 10^3$ & $238\pm 8 $& $43\pm5$\\
	B101*&Basalt-l & 0.068 & $4.21\times 10^3$ & $190\pm10 $& $43\pm5$\\
	B102*&Basalt-l & 0.068 & $5.40\times 10^3$ & $220\pm10$ & $55\pm5$\\
	\hline
	\multicolumn{6}{l}{ *Quarter-space experiments.}
%	\end{tabular}
	\end{longtable}
%	\end{center}
%	\end{table}
}
%\end{landscape}
%
%
\begin{figure*}[phtb]
	\begin{center}
	\includegraphics[width=100mm]{/Users/eri/Dropbox/2016icarus/Figures/fig11.eps}
	\caption{Crater profiles for low energy experiments. Target grains are pumice blocks with the diameter of $\sim16$ mm ((a) and (b)) and $\sim 9$ mm ((c) and (d)). (a) P502, $m_p=0.74$g, $U=69$m/s, $\xi=0.6$, (b) P501, $m_p=0.74$g, $U=183$m/s, $\xi=2.7$, (c) P301, $m_p=0.74$g, $U=122$m/s, $\xi=25$, (d) P302, $m_p=0.74$g, $U=186$m/s, $\xi=57$. Spray paint was applied in the central regions of the targets before impact experiments for (a) and (b). Note that the resulting impact craters are very shallow and rather difficult to discern without spray paint. Scale bars indicate 10 cm.}
	\label{crater-low}
	\end{center}
\end{figure*}

\begin{figure*}[phbt]
	\begin{center}
	\includegraphics[width=100mm]{/Users/eri/Dropbox/2016icarus/Figures/fig12.eps}
	\caption{Craters formed in high energy experiments and their profiles. Target grains are pumice blocks $\sim 9$ mm in diameter. (a) P203, $m_p=0.068$ g, $U=2.06$ km/s, (b) P205, $m_p=0.068$ g, $U=144$m/s, (c) P207, $m_p=0.068$ g, $U=5.05$ km/s. Note that they exhibit similar shapes to simple craters on sand targets, showing dramatic difference in crater depth in comparison with Figure \ref{crater-low} (a), (b), and (c).}
	\label{crater-high}
	\end{center}
\end{figure*}


 %%%%%%%%%%%%%
 \subsection{Crater size measurements and comparison with the classic $\pi$ scaling} \label{sec: classic pi}
 %%%%%%%%%%%%%
 %gravity scalingのはなし
The crater sizes are plotted in $\pi_2-\pi_V$ space \citep[e.g.][]{holsapple1993} in Figures \ref{pi2-pebble} and \ref{pi2-gb}, targets with angular shapes, such as basalt, pumice and sintered-glass beads, and with smooth shapes, such as glass spheres, respectively.
Here, the gravity-scaled size $\pi_2=ag/ U^2$ and the dimensionless crater volume $\pi_V=\rho_t V_c/m_p$.
The derivation of the $\pi$ parameters used this study is given in \ref{sec:piscaling}.
Because the $\pi$ parameters are based on transient crater size not final crater size after post-impact modification processes, we used the proportional relationship between the transient crater and the final crater size for simple craters: $D_c=1.18D_\text{tr}$ \citep{chapman1986,melosh1989}.

Cratering on fine-grain targets \citep{schmidt1980,mizutani1983,cintala1999} follows the gravity scaling for dry sand (solid black line in Figure \ref{pi2-pebble}).
For targets with relatively large (1--3 mm) grain targets used by \citet{cintala1999} still follows gravity scaling for dry sand, although these have much larger grains than the previous fine sand targets used by \citet{schmidt1980} and \citet{mizutani1983}.
If the mechanism of cratering on coarse-grained targets was the same as the well-studied dry sand cratering, the results should follow the line for the dry-sand gravity scaling, but many craters formed in our experiments exhibit smaller size than the sand gravity scaling line for sand by \citet{schmidt1987} in Figure \ref{pi2-pebble}.
It is noted that the crater sizes for high impact energy or high impact velocity fall almost on the gravity scaling rule for dry sand.
This result agrees with coarse-grain experiments at high velocities $\sim 5$ km/s by \citet{holsapple2014}.
In contrast, at lower impact velocities, our experimental data indicate that cratering efficiency is significantly reduced.
This might be caused by energy dissipation when the impactor hit and disrupted a target grain on the uppermost surface; material strength may play an important role in cratering under such conditions.
Thus, the classic gravity scaling may overestimate crater size on coarse-grained targets.

The same trend was observed in smooth spherical grain targets group in Figure \ref{pi2-gb}.
When the target grains are smaller than 1 mm, their crater sizes are almost the same as the fine grain targets which are several times larger than crater size on sand targets due to their small frictions.
However, cratering on 10 mm glass sphere targets in our study shows much larger crater compared with that on glass sphere targets with 10 mm and 30 mm in diameter by \citet{guettler2012}.

%ここからstrength scalingのはなし
Thus, simple application of gravity scaling cannot reproduce crater size on coarse grains; material strength must play an important role in cratering.
If cratering on coarse grains is dominated by target grain strength, non-dimensional crater volume $\pi_V=\rho_t V_c/m_p$ should be controlled by $\pi_3=\bar{Y}/ \rho_t U^2$, which is target deformation energy $\bar{Y}/\rho_t$ for unit mass scaled by impact energy $U^2$ for unit mass.
In the material strength dominated range, log $\pi_V$ should follow a straight line in $\pi_3$--$\pi_V$ space.
Because the effective strength of bulk target as granular material is combination of many factors, we cannot readily estimate or measure the effective strength $\bar{Y}$, which corresponds mainly to the bulk compression and tensile strength mainly.
Instead, here we attempt to derive a strength scaling with the effective strength for composed target grains, because if target grains are much larger than impactor size and impact energy is very small, final crater size would be close to that of a crater on a hard monolithic body ($\bar{Y}=18$ MPa) (Figure \ref{pi3-scaling}).
This might be a lower limit of crater size.
Most of our results are apparently very different from the strength scaling of monolithic rock. However, large grain size and small impact energy conditions, such as large pumice grains and large glass sphere, approaches the strength scaling.
Note that if we assume weaker effective strength, the results would shift toward smaller $\pi_3$, but the slopes do not change.
We still could not explain the steep slopes of large grain targets by changing the effective strength.
%\red{意味:基本的には材料強度則で予想されるクレーターサイズよりも桁で大きいクレータができている。ただし、衝突エネルギーが小さい時にはクレーターサイズは材料強度則で予測されるサイズに近くなっている。桁で大きいという問題はeffective strengthを変えれば平行移動することができるので、無理やり合わせようと思えば合わせられなくないが、衝突エネルギーが大きいところを無理やり合わせても傾き自体が変わることはないので、エネルギーが小さいところで急な傾きを持つことは説明できない。}

These results and analyses suggest that impact processes for high impact velocities ($> 5$ km/s) may follow the sand gravity scaling because when the impact energy is sufficiently large, energy dissipation by disrupted fragments is negligible compared with the energy to excavate the target.
In contrast, impact processes at low velocities do not follow either the gravity or strength scalings of dry sand in the classical $\pi$ scaling framework.
Compared to the gravity scaling, the cratering efficiency is reduced to less than 0.1 times the gravity scaling of dry sand.
This indicates that classic $\pi$ scaling do not account for cratering on coarse-grained surface very well. Thus, estimation of crater retention ages on sub-km asteroids with coarse-grained surfaces would require a new scaling rule for extrapolating the laboratory-scale results to real asteroid scales.

\begin{figure}[htbp]
	\begin{center}
	\includegraphics[width=80mm]{/Users/eri/Dropbox/2016icarus/Figures/fig7.pdf}
	\caption{The classical $\pi$ scaling plot with $\pi_2\pi_4^{-1/3}$ and $\pi_V$ for various grain-sized targets with angular shapes including results by \citet{schmidt1980,mizutani1983,cintala1999}, compared with the empirical scaling lines for dry sand (solid), 200\% of dry sand (gray), 50\% of dry sand (dashed), and 10 \% of dry sand (dash-dotted) \citep{schmidt1987}. If the mechanism of cratering on coarse-grained targets was same as the well-understood dry sand cratering, the results should follow the solid line, but the coarse-grained targets show smaller craters in low velocity (large gravity-scaled size $\pi_2\pi_4^{-1/3}$).}
	\label{pi2-pebble}
	\end{center}
\end{figure}

\begin{figure}[htbp]
	\begin{center}
	\includegraphics[width=80mm]{/Users/eri/Dropbox/2016icarus/Figures/fig8.pdf}
	\caption{The classical $\pi$ scaling plot with $\pi_2\pi_4^{-1/3}$ and $\pi_V$ for various grain-sized targets with smooth shapes including results by \citet{schmidt1980,yamamoto2006,guettler2012}, compared with the empirical scaling lines for dry sand (solid), 200\% of dry sand (gray), 50\% of dry sand (dashed), and 10 \% of dry sand (dash-dotted) \citep{schmidt1987}. The results do not follow the scaling line.}
	\label{pi2-gb}
	\end{center}
\end{figure}

\begin{figure}[htpb]
	\centering
	\includegraphics[width=80mm]{/Users/eri/Dropbox/2016icarus/Figures/fig9.pdf}
	\caption{The classic $\pi$ scaling plot with $\pi_3$ and $\pi_V$ assuming the effective strength of $\bar{Y}=18$ MPa, the strength of rigid rocks. The results from this study and previous experiments \citep{schmidt1980,mizutani1983,cintala1999,yamamoto2006,guettler2012,holsapple2014} are compared with the results for hard rock (solid) and cohesive sand (dotted) \citep{schmidt1987}. For most of cases, the crater sizes are much larger than what the strength scaling on hard rock predicts. However, when a grain size is much larger than a impactor size, the crater size is comparable to the strength scaling on hard rock (solid line).}
	\label{pi3-scaling}
	\centering
\end{figure}

%------------------------------------------------------------%
\subsection{Quarter-space experiments}\label{sec:observation}
%------------------------------------------------------------%

The above experimental results suggest that the classic $\pi$ scalings (gravity or strength scaling) are not capable of describing accurately cratering on coarse-grained targets with armoring effect. This suggests that accurate reproduction of crater size on coarse-grained targets would require a new scaling law based on more in-depth consideration on cratering processes targets.
Thus, we conducted quarter-space experiments to observe cratering processes underneath the target plane in detail to obtain hints for armoring processes.
Quarter-space experiments have been very effective in observing physical processes in impact cavities immediately after high-speed collision under a variety of conditions \citep[e.g.][]{pietukowski1980,schultz2015}.
Figure \ref{snapshots} shows time-series images of a polycarbonate projectile impacting a pumice target at $\sim 4.3$ km/s (P206).
Note that a pre-impact image was subtracted from each image in order to visualize moved grains.
Upon the contact between the impactor and a target grain at a high velocity, flash of light occurs first because of extremely high shock pressure due to solid-to-solid collision \citep[e.g.][]{sugita1998}.
As shock wave propagates, the very fast ejecta from the first-contact point emerges.
It is noted that the neighboring grains did not move much at first moment.
The shock wave in a coarse-grained target is far slower than that within the rigid targets which usually have shock wave velocity on the order of km/s.
Because fragments in the very fast ejecta are fully fractured, they do not have size large enough to be measured.
We could see that the fragmentation of target grains occurred in the very vicinity of the impact site, agreeing with the simulation results by \citet{barnouin2002}.

After the fracturing stage, the momentum transfer triggers subsequent excavation flow.
The unfractured grains move along the excavation flow in a very similar fashion as the flow of the gravity-dominated simple crater.
However, the excavation flow speed was much slower than the excavation flow in fine-grained sand targets, resulting in much smaller final crater size.
The excavation stage appears virtually uninfluenced by the material strength of constituent grains but might be controlled by friction between grains as dry sand.
This is probably because target grains were not fractured in the flow field, the material strength of individual grains does not play an important role here. %Microscopically they might grind each other and the small fractures of edges can be occurred, but the slow excavation flow cannot generate high pressure to fracture target grains heavily.

These observations suggest that the material strength of constituent grains may influence only the compression and fracturing stage of cratering.
However, the very first stage of cratering, where momentum is transferred from the impactor to the first-contact grain, is influenced directly by mechanical strength of target grains.


\begin{figure*}[phtb]
	\centering
	\includegraphics[width=120mm]{/Users/eri/Dropbox/2016icarus/Figures/fig10.eps}
	\caption{Background-subtracted time-series of a polycarbonate ($a=2.38$ mm) impact on a pumice target with $\sim 9$ mm in mean diameter at $\sim 4.3$ km/s of  velocity (P206). Because these images are background-subtracted images, grains that have not moved from the pre-impact conditions are not appear in this images. White doted line indicates the surface of a pre-impact target and colored doted lines depict shock fronts moving outward. Different colors indicate different moments. At the very early stage, target grains are fractured but not excavated (0 -- 2.6 $\mu$s). The last image shows the excavated area at the moment and the final crater area.}
	\label{snapshots}
	\centering
\end{figure*}

%\begin{figure}[htbp]
%	\begin{center}
%	\includegraphics[width= 100mm]{/Users/eri/Dropbox/2016icarus/Figures/Chap2/xi_vs_aspect_ratio.eps}
%	\end{center}
%	\caption{The aspect ratio $D/d$ of craters of Pumice-m and sintered-glass-beads targets. The aspect ratio for simple craters are commonly 1/3 to 1/4.}
%	\label{xi_aspect}
%\end{figure}

%

%%%%%%%%%%%%%%%%%%%%%%%%%%%%%%%%%%%%%%%%%%%%%%%%%%%%%%%%%%%%%%%%%%%%%%%%%%%%%%%%%%%%%%%%%%%%%%%%%%%
\section{Discussions}
%%%%%%%%%%%%%%%%%%%%%%%%%%%%%%%%%%%%%%%%%%%%%%%%%%%%%%%%%%%%%%%%%%%%%%%%%%%%%%%%%%%%%%%%%%%%%%%%%%%

%------------------------------------------------------------%
\subsection{Extension of the $\pi$ scaling} \label{sec:extension-pi}
%------------------------------------------------------------%
The above results in this study and previous experimental studies (Figure \ref{phase}) strongly suggest that when the impact energy is not sufficiently large the crater volume on the coarse-grained targets would be influenced noticeably by target disruption energy $Q_D^*$ for unit mass \citep[e.g.][]{fujiwara1980b, benz1999}, target material density $\delta_t$ and the target grain radius $r_t$.
Thus, we attempt to derive a new $\pi$ scaling, starting from the following equation instead of eq. \eqref{pi-start-eq}:
\begin{equation}
	V_c=f^*[a,\,U,\,\rho_p,\, \rho_t,\,Y,\, g,\, Q_D^*, \,\delta_t,\, r_t]. \label{pinew-start-eq}
\end{equation}
Note that $\rho_t=\phi \delta_t$, where $\phi$ is the bulk porosity of target.
Then, the other three dimensionless parameters in addition to the classic $\pi$ parameters are automatically derived by dimensional analysis:
\begin{align}
	\pi_5&=\frac{Q_D^*}{U^2},\\
	\pi_6&=\frac{r_t}{a}=\frac{D_t}{D_p},\\
	\pi_7&=\frac{\delta_t}{\rho_p}.
\end{align}
Furthermore, the ratio $\xi$ of impact energy to the disruption energy of a target grain and the size ratio $\psi$ of projectile to target grains are given as the combinations of these $\pi$ parameters:
\begin{align}
	\xi&=\frac{1}{2}\pi_5^{-1}\pi_6^{-3}\pi_7^{-1},\\
	\psi&=\pi_6^{-1}.
\end{align}
Because the dimensional analysis indicates that the phenomena can be parameterized by those seven parameters if the phenomena is controlled by the physical properties in eq. \eqref{pinew-start-eq}, it is possible that the grain disruption energy ratio $\xi$ and the projectile target grain size ratio $\psi$ are the main controlling parameters that make cratering processes on coarse-grained targets.

The observations of the quarter-space experiments in Sec.\ref{sec:observation} suggest that the fracturing stage and the excavation stage should be considered separately.
More specifically, because the flow in the excavation stage is similar to that for simple-crater formation, modification of the energy and momentum terms in the point source for the $\pi$ scaling formulation may account for the difference in cratering between coarse-target cratering and cohesionless-sand cratering.
As suggested by \citet{guettler2012}, the final crater diameter might be influenced by the mass ratio of the impactor to the target grain through the momentum conservation law during impact.
Note that kinetic energy does not necessarily conserve during impact because initial impact energy can be lost through various dissipation processes.
Assuming that the first-contacted target grain moves together in a group even after fractured, the velocity $U^*$ of the target grain and the impactor after the collision would be
\begin{align}
	U^*&=\frac{(1+\epsilon)}{1+m_t/m_p}U=\frac{(1+\epsilon)U}{1+\pi_6^3\pi_7}=\Pi U,\\
	&{\rm where}\quad \Pi=\frac{m_p}{m_p+m_t}=\frac{1+\epsilon}{1+\pi_6^3\pi_7},\notag
\end{align}
where $\epsilon$ is the coefficient of restitution.
When the impacted target grain and the impactor move together after the collision, the coefficient of restitution is 0 (i.e., a perfectly inelastic collision).
We can assume that $\epsilon \sim 0$ because the first contact grain would experience highly inelastic deformation upon impact.
A similar dimensional analysis leads to the extended $\pi$ scaling for coarse-grained target expressed as \eqref{comp-new-pi}:
\begin{align}
	\pi_V^*&=\frac{\rho_t V_c}{(m_p+m_t)}=K_1\left[\pi^*_2\pi_4^{-1/3}+K_2\xi^{-(2+\mu_2)/2}+K_3\pi_3^{*(2+\mu_3)/2} \right]^{-3\mu_1/(2+\mu_1)} \label{comp-new-pi}\\
	%\mathcal{F}^*[C^*(\Pi),\pi_2^*(\Pi),\pi_3^*(\Pi),\xi(\pi_5,\pi_6,\pi_7),\pi_4], \notag\\
	\intertext{where,}
	\pi_2^*&=\frac{a^*g}{U^{*2}}=\Pi^{-7/3}\pi_2, \notag\\
	 \pi_3^*&=\frac{\bar{Y}}{\rho_tU^{*2}}=\Pi^{-2}\pi_3, \notag\\
	 \xi&=\frac{m_pU^2/2}{m_tQ_D^*}=\pi_5^{-1}\pi_6^{-3}\pi_7^{-1}/2 \notag\\
	 \pi_4&=\frac{\rho_t}{\rho_p}.\notag
\end{align}
Note that the $\pi_V^*,\,\pi_2^*$ and $\pi_3^*$ have the same forms as eqs.\eqref{piV}--\eqref{pi3}, but here we use effective size $a^*$ and effective velocity $U^*$. Besides the gravity scaling term and the strength scaling term, eqs.\eqref{comp-new-pi} contain another strength term that depends on $\xi$. Both the second and the third terms indicate the influence of strength, but they reflect different types of strengths. More specifically, the second one indicates material disruption strength of constituent grains and the third one indicates bulk strength, respectively.

Using these newly derived non-dimensional parameters, we obtain relation between crater efficiency and dimensionless parameters modified for coarse grain disruption in $\pi_2^*-\pi_V^*$, $\pi_3^*-\pi_V^*$ and $\xi - \pi_V^*$ spaces (Figures \ref{newpi2}--\ref{xi_vs_newpiV}).
We compared crater efficiency $\pi_V^*$ between experimental data and our new scaling law as a function of the ratio of impact energy to gravity potential (Figure \ref{newpi2}), impact energy to bulk break-up energy(Figure \ref{newpi3}), and the impact energy to disruption energy of a target grain(Figure \ref{xi_vs_newpiV}).
We can assess which scaling law express the experimental data better based on these comparisons.

Figure \ref{newpi2} compares the new gravity scaling laws and our experimental data. Figure \ref{newpi2}(a) shows the results for targets with angular-shaped grains, and Figure \ref{newpi2}(b) shows targets with smooth spherical grains.
Both results indicate non-dimensional crater volume $\pi_V^*$ at high velocities (i.e., small $\pi_2^*$) are fitted well with the new gravity scaling.
Our data for the large crater efficiency $\pi_V^*> 500$ for angular-grain targets and $\pi_V^*>10$ for spherical-grain targets are mostly distributed rather tightly around the gravity scalings of dry sand (i.e., 100\%) and glass beads (i.e., 200\%), respectively, within a factor of 2.
However, that crater efficiency data smaller than the above $\pi_V^*$ values do not follow the gravity scaling trend (i.e., slope of $\sim -0.5$).

In contrast, Figure \ref{xi_vs_newpiV} shows that those at low velocities (i.e., large $\pi_2^*$) are expressed well with the disruption strength scaling.
It should be also noted that sintered-glass-beads targets have angular shapes but follow the gravity scaling for smooth spherical grains more closely in Figure \ref{newpi2}.
This is possibly because these targets behave as smooth glass beads when it is completely fractured owing to the very weak inter-grain bonding.
With the new scaling laws, most of the data for coarse grained targets can be fitted with error less than a factor of two (Figure \ref{check-new-pi}), which is much better than the classic scaling laws in Sec. \ref{sec: classic pi} (Figures \ref{pi2-pebble}-\ref{pi3-scaling}).

At least under our experimental conditions, the crater size can be expressed with only two term: the new gravity scaling term expressed with $\pi_2^*$ and the new disruption strength scaling term expressed with $\xi$.
Although the break-up strength $\pi_3^*$ should have potential to account for another experimental trend, such as less energetic impacts, it turned out not to fit our experimental data better.
Thus, our experimental data can be fitted best with the combination of the new gravity term expressed with $\pi_2^*$ and the new disruption strength term expressed with $\xi$.
\begin{align}
	\pi_V^*&=K_1\left[\pi_2^{*}\pi_4^{-1/3}+K_2\xi^{-(2+\mu_2)/2}\right]^{-3\mu_1/(2+\mu_1)}. \label{new-pi}
\end{align}
where, $K_1=0.24 $, $\mu_1=0.41$ and $K_2=0.01$, $\mu_2=0.54$ for angular-shaped grain targets, and $K_1=0.6 $, $\mu_1=0.41$ and $K_2\sim 0$ for smooth spherical grain targets.
The third term in eq. \eqref{comp-new-pi} can be omitted.
In other words, intrinsic material strength of individual grains do not influence much when $\xi>1$.
When impact energy is too small, i.e., $\xi<1$, only one or few target grains are fractured or one target grain is cratered \citep{guettler2012}.
In such cases, crater size would follow the intrinsic material strength of individual grains, such as basalt and soda glass.
However, when the energy ratio $\xi$ is larger than unity, the experimental results can be expressed for two different regimes. 
Consequently, the full equation eq.\eqref{new-pi} is not always needed to describe the crater size. Only one term in eq.\eqref{new-pi} can predict the crater size accurately under appropriate conditions. 
More specifically, in the new gravity regime,
\begin{align}
	\pi_V^*&=K_1\pi_2^{*-3\mu_1/(2+\mu_1)}, \label{newgravity}
%	V_c&=K_1\left(\frac{m_p+m_t}{\rho_t}\right)\left(\frac{ga^*}{U^{*2}}\right)^{-3\mu/(2+\mu)}\left(\frac{\rho_t}{\rho_p}\right)^{\mu/(2+\mu)}, \notag\\
%	&=K_1\left(\frac{m_p}{\rho_t}\right)\Pi^{(6\mu-2)/(2+\mu)} \pi_2^{-3\mu/(2+\mu)}\pi_4^{\mu/(2+\mu)}.\label{newgravity}
\end{align}
and in the new fracture strength regime,
\begin{align}
	\pi_V^*&=K_1K_2\xi^{3\mu_1(2+\mu_2)/2(2+\mu_1)}. \label{newstrength}
%	V_c&=K_1\left(\frac{m_p+m_t}{\rho_t}\right)\left(\frac{1/2m_pU^2}{m_tQ_D^*}\right)^{3\mu_2/2}, \notag\\
%	&=K_1\left(\frac{m_p}{\rho_t}\right)\Pi^{??}\xi^{3\mu_2/2}.\label{newstrength}
\end{align}
\red{They are switched by the magnitude relation of $\pi_2^*$ and $\xi$, i.e., the new gravity scaling is applicable, when the velocity is large enough.}
It is noted that as $m_p(\gg m_t)$ increases the above eq.\eqref{new-pi} approaches the classic gravity $\pi$ scaling discussed in \ref{sec:piscaling}.

These situations of the cratering are summarized in Figure \ref{psi-xi}, in which the applicable ranges of conditions for this new scaling (eq.\eqref{new-pi}) is shown. Note that our scaling law cannot apply for $\xi<1$ and $\psi<1$ where craters are much smaller than what the new scaling law predicts.
This is because as the energy is smaller than the disruption energy of one target grain and the crater size would follow the classic strength scaling.
The way armoring effect works differ greatly depending on size ratio $\psi$ and disruption energy ratio $\xi$.
Based on which scaling law works, we can classify the $\xi$--$\psi$ space into three regimes.
(1) No armoring effect (the classic gravity regime): $\psi>1$ or $\xi$ is sufficiently large, (2) armoring regime I, where the new crater scaling law can be applied to predict reduced size craters: $\psi<1$ and $\xi \gtrsim 1$, (3) armoring regime {\II} , where cater size is smaller than what our scaling law predicts: $\psi<1$ and $\xi<1$.
Such conditions of impacts may leave no apparent topographic craters. Within the armoring regime \II, there might be a transition from one grain disruption to the classic strength regime, in which a small ``intra-grain'' crater is formed. Although such processes may be interesting, detailed characterization is  beyond the scope of this study.

\begin{figure*}[tbp]
	\centering
	\includegraphics[width=80mm]{/Users/eri/Dropbox/2016icarus/Figures/fig13.pdf}
	\caption{The new scaling law in the $\pi_2^*- \pi_V^*$ space, compiling our data the previous data by \citet{schmidt1980, mizutani1983, cintala1999, yamamoto2006, guettler2012, holsapple2014}. The solid black lines indicate classic gravity scaling law for dry sand. Gray line, dotted line, and one-dot line exhibits lines of 200\%, 50\%, and 10\% of sand case, respectively. Different shapes of grains are compared: (a) Target with angular grains and (b) smooth spherical grains.}
	\label{newpi2}
	\centering
\end{figure*}
\begin{figure}[htbp]
	\centering
	\includegraphics[width=80mm]{/Users/eri/Dropbox/2016icarus/Figures/fig14.pdf}
	\caption{The new scaling law in the $\pi_3^*- \pi_V^*$ space, compiling with the previous data by \citet{schmidt1980, mizutani1983, cintala1999, yamamoto2006, guettler2012, holsapple2014}. We assumed that the effective strength of $\bar{Y}=18$ MPa, which is similar to hard rocks. The experimental results are compared with the results for hard rock (solid) and cohesive sand (dotted) \citep{schmidt1987}.}
	\label{newpi3}
	\centering
\end{figure}

\begin{figure}[htbp]
	\centering
	\includegraphics[width=80mm]{/Users/eri/Dropbox/2016icarus/Figures/fig15.eps}
	\caption{Experimental results in $\xi$-$\pi_V^*$ space, compiling with the previous data by \citet{schmidt1980, mizutani1983, cintala1999, yamamoto2006, guettler2012, holsapple2014}. When the energy ratio $\xi$ is small but greater than unity, crater size follows the new scaling law. When impact energy is large enough, the new scaling law switches from the disruption scaling to the new gravity scaling. The new scaling lines are shown by solid lines and dash-dotted lines are 50\% and 200\% of each new scaling value. (a) Target with angular-shaped grains: the data cluster around two lines: one is pumice and basalt block targets (black solid) and another is sintered-glass-beads targets (gray dashed) in low energy ratio. (b) Target with smooth spherical grains: the data are cluster around one line (black solid) in low energy ratio. When the energy ratio $\xi<1$, only few grains were shuttered or a crater hole was made on a grain \citep{guettler2012}.}
	\label{xi_vs_newpiV}
	\centering
\end{figure}

\begin{figure}[htbp]
	\centering
	\includegraphics[width=80mm]{/Users/eri/Dropbox/2016icarus/Figures/fig16.pdf}
	\caption{Comparison between Experimental results and predicted $\pi_V^*$. Experimental data in this study and that by \citet{schmidt1980, mizutani1983, cintala1999, yamamoto2006, guettler2012, holsapple2014} are compared. If the new scaling predicts the experimental data acculately, symbols would cluster along the 45$^\circ$ line (dashed line). Both our results and previous results follow mostly within a factor of 2, although there are some symbols slightly out of the factor of 2. The region for a factor of 2 is shown with gray hatch between the ratio of experimental values to predicted values from 0.5 and 2. (a) Target with angular-shaped grains and (b) target with smooth spherical grains. When the non-dimensional volume $\pi_V^*<10$ (below the vertical line), symbols are scattered and do not follow the central 45$^\circ$ lines.}
	\label{check-new-pi}
	\centering
\end{figure}


\begin{figure}
	\centering
	\includegraphics[width=80mm]{/Users/eri/Dropbox/2016icarus/Figures/fig17.pdf}
	\caption{A phase diagram for armoring effect. The conditions of experiments and the degree of armoring effect are shown. Filled symbols indicate substantial reduction in cratering efficiency following the new scaling (i.e., eq.\eqref{new-pi}). Open symbols indicate the cratering efficiency following the classical gravity scaling. Small symbols are crater size smaller than what our scaling predicts. Armoring effect by target grain disruption were observed in the magenta region. Our crater size scaling is not applicable for $\xi<1$ and $\psi<1$ (blue hatch). In this range, a crater is formed on one target grain following the classic strength scaling or the first contacted grain is simply disrupted, leaving no apparent crater.}
	\label{psi-xi}
	\centering
\end{figure}

%------------------------------------------------------------%
\subsection{Effect of neighbor grain fragmentation}\label{sec:mechanism}
%------------------------------------------------------------%
Because shockwaves travel at velocities much higher than subsequent excavation flow, in initial shock fragmentation due to a collision between a projectile and target grains would complete before substantial excavation motion takes places.
%Thus, the momentum/energy transfer from the projectile to the target grains may not interfere directly with the excavation flow but simply influence the cratering efficiency, resulting in a similar crater shape and different crater sizes.
Thus, even if the impact energy is same, the fashion that cratering efficiency is influenced by such momentum/energy transfer from the projectile to the target grains can be interpreted well when the degree of fragmentation of target grains is taken into account.

When target grain size is large and impact energy is small, the dominant fraction of impact energy given to the first-contact grain will be carried by shockwave producing high-speed fragments from the first contact grain.
Such shockwave will be reflected efficiently on the surface of the grain and only small portion of the compression wave transfer to the neighbors through contacting points, resulting in preventing strong compression wave traveling farther.
Thus, the first-contact grain is disrupted but surrounding grains are not disrupted at low impact energies. In this case, the impact energy is largely consumed by fragmentation and fragment ejection process dissipated by ejecta from the first-contact grain.
The momentum cannot be carried to the surrounding grains because there are only small surface areas of contact points (armoring regime I).
When impact energy becomes larger, the impact energy will disrupt surrounding grains as well, but the fragments from sub-surface grains will not escape to space because fragments and grains in the upper layer behave as a lid.
Thus, more energy is partitioned into the kinetic energy of excavation flow of target grains rather than ejecta from the first-contact grain (armoring regime I).
When the impact energy is sufficiently large, the amount of high-speed fragments from the first-contact grain increases.
By colliding the surrounding grains, these fragments transfer the momentum and kinetic energy efficiently to the surrounding grains.
As a result, full excavation flow will develop, and a crater as large as a crater on sand (i.e., gravity regime) will form.
%The large scatter in crater size for impact energy comparable to the disruption energy of one grain (i.e., $\xi\sim1$) may be due to the transition from armoring regime I to armoring regime \II.

There are also a series of experiments yielding results supporting that the energy transfer from the first contact grain to the surrounding grains is influenced greatly by the geometry of target grains.
\citet{durda2011} conducted impact experiments to both a block resting on sand and a block partially buried in sand.
Even when impact energy was comparable to the catastrophic disruption of the target blocks, the damage of a buried block was much less than a block rested on a sand surface.
Furthermore, a block rested on a sand surface was not fully disrupted though a block suspended in space would be catastrophically disrupted at this level of impact energies.
As the surface area of the block touching the sand increases, the damage decreases.
This observation supports that disruption impact energy could be passed to surrounding sand grains through numerous contact points.
Thus, the energy carried away from the first contacted grain to the neighbors would be affected by the number of contact points.
For the case of coarse-grained targets investigated in this study, the fragments may play the role of the fine sand in the experiments by \citet{durda2011} and carry the momentum and energy efficiently to neighboring grains.
%However, this energy leak would happen only when the impact velocity is comparable to the material sound speed.

\begin{figure}[htbp]
	\centering
	\includegraphics[width=120mm]{/Users/eri/Dropbox/2016icarus/Figures/fig18.eps}
	\caption{The schematic images of transition from armoring regimes to gravity regime as impact energy increases. Armoring regime \II: When impact energy is comparable to the disruption energy of a target grain, crater size is dominated by material strength (left). Armoring regime I: When impactor size is comparable to target grain size and the impact energy is larger than disruption energy of a target grain, the crater size is dominated by grain disruption strength (middle, magenta line in Figure \ref{Itokawa-crater}(a)). Gravity regime: When the impact energy is sufficiently larger than the disruption energy of a target grain, the crater size is dominated by momentum transfer like granular material (right, blue line in Figure \ref{Itokawa-crater}(1)).}
	\label{schema-energy}
	\centering
\end{figure}

 %%%%%%%%%%%%%%%%%%%%%%%%%%%%%%%%%%%%%%%%%%%%%%%%%%%%%%%%%%%%%%%%%%%%%%%%%%
\section{Planetary Implications}\label{sec:implication}
 %%%%%%%%%%%%%%%%%%%%%%%%%%%%%%%%%%%%%%%%%%%%%%%%%%%%%%%%%%%%%%%%%%%%%%%%%%
In this section, we discuss a couple of important implications of our newly obtained scaling law for impact cratering on coarse-grained targets.
First, we discuss the crater retention age on Itokawa in comparison with geochemical age measurements of Itokawa samples returned from by the Hayabusa spacecraft to the Earth.
Then we further discuss mass loss rate for small rubble-pile asteroids, such as Itokawa, Ryugu, and Bennu, and implications for long-term evolution of the asteroid main belt.

%------------------------------------------------------------%
 \subsection{Implication for Itokawa crater retention age}\label{sec:itokawa-age}
%------------------------------------------------------------%
It is rather natural to assume that crater retention age on asteroid surface covered with large boulders with significant strength can be estimated with strength crater scaling. \citet{michel2009} showed that strength scalings with both basalt and soft rock strength values would yield relatively old surface ages (75 Myr -- 1 Gyr) for Itokawa if the crater candidates found by \citet{hirata2009} are of impact origin.
 However, Itokawa is most likely a rubble-pile asteroid, which is weakly gathered with gravity and cohesion.
 Our experiments suggest that the crater size was much greater than the crater size on a monolithic body, although the armoring may reduce crater size significantly depending on the disruption energy ratio $\xi$.
 Whenever impact energy is large enough to catastrophically disrupt the first-contact boulder, our results would predict that crater size on Itokawa would be in between the gravity scaling for sand and the soft rock strength scaling. This suggests that the real Itokawa surface age would be much younger than values based on the classical strength scaling.
Because the crater retention age estimate may be influenced greatly by crater scaling law, we revisit the Itokawa crater retention age estimate using our armoring crater scaling law based on the distribution of the larger crater-like features $>100$ m, which are largest features on Itokawa.

For Itokawa, we used the bulk density of Itokawa $\rho_a=1900$ kg/m$^3$ \citep{fujiwara2006}, typical impact velocity $U=5.3$ km/s in the main belt \citep{bottke1994}, and the surface gravity $g=7.5\times 10^{-5}$m/s$^2$ \citep{tancredi2015} of Itokawa for our calculation.
Note that an impactor $\sim 1.8$ m in diameter can catastrophically disrupt a 40 m-diameter boulder, the maximum diameter for Itokawa boulders, at the assumed impact velocity.
In other words, impactors larger than 1.8 m could always form topographically well-defined craters.
When we see a crater on Itokawa, that indicates the collision $\xi>1$ happened, because it would be difficult to discern a crater with $\xi<1$.
We take the characteristic boulder size of 2 m because $P(>2 \text{m})\sim 0.5$ with the SFD condition of boulders on Itokawa (\ref{boulder-dist}), which means the probability for impactors to hit the surface boulders larger than 2 m is approximately 50 \% on Itokawa when $D_\text{min} \sim 0.1$ m.
Assuming for simplicity in calculation that the Itokawa surface is covered with boulders uniformly with the characteristic size $\sim2$ m, we obtain the crater efficiency from eq.\eqref{new-pi},
\red{In this case, 100-meter-sized craters could be formed in the range $\xi>1$ and we can apply the new scaling for the surface age estimate of Itokawa.}
%\begin{equation}
%	\chi_V=\begin{cases} (1+(D_t/D_p)^3)^{-(6\mu-2)/(2+\mu)}& (D_p\gtrsim 1.8\, {\rm m}) \\ (1+(D_t/D_p)^3)^{-3\mu+1} &(0.04\, {\rm m} \lesssim D_p \lesssim {1.8} \,{\rm m} ),
%	\end{cases}\label{eq:itokawa-crater}
%\end{equation}
where projectile density is comparable to target bulk density, i.e., $\rho_p\sim\rho_t$.
%For simplicity assuming the minimum cratering efficiency was $\chi_D^g=0.5$ when the craters were morphologically identified, the crater sizes could be estimated between the minimum and maximum efficiency(Figure \ref{Itokawa-crater}). It should be emphasized that even if the armoring effect works, the crater size is much larger than strength scaling and closer to the gravity scaling of the dry soil.
%
%
The probability of collisions on an asteroid is
\begin{align}
	p_\text{coll}&=\frac{P_I(D_t+D_p)^2N_\text{pairs}}{4 N_\text{total}}, \notag\\
	&=\frac{P_I(D_t+D_p)^2N_pN_t}{4(N_t+N_p)},
\end{align}
where $N_\text{pairs}$ is the number of target--projectile pairs and $N_\text{total}$ is the total number of targets and projectiles.
The intrinsic collision probability for MBA--MBA collisions, that for NEA--NEA collisions, and that for NEA--MBA collisions are $2.86\times 10^{-18}$ km$^{-2}$ yr$^{-1}$, $15.3\times 10^{-18}$ km$^{-2}$ yr$^{-1}$ and $2.18\times 10^{-18}$ km$^{-2}$ yr$^{-1}$, respectively \citep{bottke1994}.
The number of target--projectile pairs of NEOs is much smaller than MBAs because the number of NEAs are $1/1000$ times that of MBAs.
Although Itokawa is in a near-Earth orbit now, where the intrinsic collisional probability and the impact velocity are higher than in the main belt,
the crater production rate is $\sim 50$ times lower on an NEA than on an MBA because of the much higher number density of potential projectiles in the main belt.
In other words, it takes $\sim 50$ times longer to form the same crater distribution in the near-Earth orbits.
Because of this low crater production rate in a near-Earth orbit, cratering in a near-Earth orbit may not account for the observed crater population on Itokawa.
Thus, the cratering history on Itokawa possibly reflects the main-belt impacts because of the short dynamical lifetime (i.e., several million years) in the near-Earth orbits.
Assuming that Itokawa is a spherical body with $D_\text{Ito}=0.326$km, the frequency $p_\text{Ito}(D_p)$ that the projectile with $D_p$ in diameter hits Itokawa is
\begin{equation}
	p_\text{Ito}(D_p) = P_I\frac{(D_\text{Ito}+D_p)^2}{4}n_a(D_p),
\end{equation}
where $n_a(D)$ is the number of MBAs with diameter $D$, or $n_a(D)=-{\rm d} N_a(>D)/{\rm d} D$ by definition.
Then the possibility $\mathcal{P}_\text{Ito}(>D_p)$ that the projectile larger than $D_p$ in diameter hits Itokawa is,
\begin{align}
	\mathcal{P}_\text{Ito}(>D_p) &= P_I\frac{(D_\text{Ito}+D_p)^2}{4}N_a(>D_p) \notag\\
	&\sim \frac{P_I D_\text{Ito}^2N_a(>D_p)}{4},\qquad (\because\quad D_\text{Ito} \gg D_p).
\end{align}
Based on numerical simulation, \citet{obrien2005} estimated the the size distribution of small MBAs, because they are too small and faint to be observed directly from the Earth;
\begin{equation}
	N_a(>D)=1.85\times 10^{13}D^{-2.91},\qquad (1<D<100\text{m}).
\end{equation}
Thus, the average time interval that the impactor larger than $D_p$ hits Itokawa can be calculated as $\Delta \tau =1/\mathcal{P}_\text{Ito}(>D_p)$;
\begin{equation}
	\Delta \tau(>D_p) = \frac{4}{P_ID_\text{Ito}^2N_0 D_p^{\alpha}}.
\end{equation}
Here, impactor size $D_p$ can be estimated from observed crater size using the non-linear relationship by eq. \eqref{new-pi} (Figure \ref{Itokawa-crater}).
It is noted that the age estimate from crater distribution is very sensitive on the impactor distribution and would be subject to the future revision in the size distribution of asteroids. However, the framework we propose here for crater retention age estimate for coarse-grained surfaces will change due to such revision.

Larger craters could be retained for longer time; smaller ones are much more subject to modification and obliteration processes, such as seismic shaking \citep[i.e.,][]{michel2009}.
Thus, larger craters would be a more reliable measure for crater retention age on small asteroids, such as Itokawa.
When such obliteration processes are not taken into account, obtained crater retention age would give a lower estimate for the asteroid surface age.
This age is still important because crater obliteration processes on small asteroids have not been quantitatively assessed yet.
On the Itokawa surface, five largest circular depressions with $>100$ m in diameter were observed \citep{hirata2009}.
Because these circular depressions exhibit a number of characteristics as impact craters, we assume that they are of impact origin in the following discussion.
A crater $\sim 100$ m rim-diameter would require a projectile diameter $D_p\sim 1.7$ m, which corresponds to $\Delta \tau (>100 \rm{m})\sim 3.3$ Myr.
Assuming that the possibility of meteorite impacts can be expressed with a Poisson distribution, we can estimate error due to statistical fluctuation associated with a small number (i.e., five). We could estimate the posterior probability distribution function based on Bayesian statistics assuming that the prior probability distribution function of how many impact events would occur is uniform (i.e., constant) independent of the number of impact events.
This assumption on the probability is reasonable because there is no observational bias to observe an asteroid depending on the number of craters on its surface.
When there are five craters observed, the average impact frequency for the period when the observed five craters were formed is between three to ten with 90\% confidence.
Thus, the time to accumulate such five large impact craters without any crater obliteration process is estimated to be 10 -- 33 Myr (90\% confidene).

If these five observed craters were formed in the near-Earth orbit, it would have taken more than 500 Myr due to much less encounters with other asteroids in the near-Earth orbit as we discussed above.
However, \citet{michel2005} calculated the lifetime in the current orbits and found that the median lifetime is 8.3 Myr and that only 4 out of 39 cases could survive more than 100 Myr.
Thus, the possibility that all the five 100 m-sized craters were formed in the near-Earth region is less than 10\%.
Consequently, it is highly likely that a large fraction of the large crater-like depressions were formed in the main belt and this crater retention age may reflect the resurfacing  time of the asteroid in the main belt.
\begin{comment}
This has an important implication for resurfacing on Itokawa in its near-Earth orbit.
Although many NEAs, such as Q-type asteroids, may experienced major resurfacing presumably due to tidal force upon close encounters to terrestrial planets ({binzel2010}), Itokawa probably did not experience such extensive resurfacing after it entered its near-Earth orbit.
Otherwise, its crater retention age would be much younger.
\end{comment}
%In other words, even if there were tidal effect during possible encounter(s) with Mars and/or Earth, the tidal force have not been able to obliterate the craters \red{$\sim 100$ m in diameter}.
%Otherwise, Itokawa possibly did not experience any encounter with terrestrial planet involving global resurfacing after coming to the current orbit.

%------------------------------------------------------------%
 \subsubsection{The depletion of small circular depressions on Itokawa}
%------------------------------------------------------------%
\red{The number density of large circular depressions ($\gtrsim 100$ m) on Itokawa are comparable to that of the Moon.
However, the cumulative size distribution of small circular depressions on Itokawa has a significantly shallower slope than that of small craters on the Moon.
Figure \ref{Itokawa-crater} shows that all the circular depression on Itokawa influenced by armoring effect.
The slope of the new scaling law changes around $\sim 30$ m in diameter; the crater size of 30 m to 100 m has steeper slope than that of several meters to 30 m.
The slope of scaling law is steeper, the cumulative distribution of formed craters is shallower. 
Thus, if the armoring effect contributes the crater distribution on Itokawa dominantly, the number distribution of small circular depressions ($< 30$ m) should have steeper slope than larger ones.
However, the observed number distribution of small circular depressions has shallower slope than larger ones.
This observation implies that the number distribution of small craters are greatly controlled by obliteration processes, such as regolith convection and migration.
}
%------------------------------------------------------------%
\subsection{Comparison with other Itokawa age estimates}\label{sec:itokawa-sample}
%------------------------------------------------------------%
Itokawa is the only planetary body from which the samples with geologic context are returned other than the Moon. These samples allow a variety of geochronological analyses for surface exposure ages of Itokawa. Thus, it is very important to compare the geochemical ages with our crater retention age. Some of the ages exhibit values similar to our crater retention age, which reflect the resurfacing timescale for $\sim$10 m of depth. It is not clear, however, what this coincidence actually means. In this section, we review the ages based on both samples and remote-sensing data and discuss implications for surface evolution of Itokawa.

For example, our crater retention age estimate (10 -- 33 Myr) based on the new scaling law for coarse grains is much younger than the previous age estimate (75 Myr -- 1 Gyr) based on the strength scalings for rigid bodies \citep{michel2009}. 
These different estimates can be examined with the results from the geochemical analyses of samples returned from Itokawa, such as spectral ages, cosmic ray exposure (CRE) ages, rim thickness ages, Ne-implantation ages, and solar flare track distribution ages. Various ages of Itokawa are summarized in Figure \ref{itokawa-ages} as a function of characteristic depth. For example, the CRE ages, which reflect the residence time of several meters deep, were found $<$ 8 -- 66 Myr \citep{nagao2011} and $\sim$1.5 Myr for Ne \citep{meier2014}, and $>$3 -- 4 Myr for Be \citep{nishiizumi2015}, and the visible-near-infrared (VNIR) spectral ages, which reflect the exposure time of uppermost surface, were found $<$10 Myr \citep{koga2014} and $\sim$2 -- 8 Myr \citep{bonal2015}.

First, it should be noted that the spectral ages and the CRE ages are estimated under the near-Earth condition, where the solar wind flux is about 4 times that in the main belt. Depending on the ratio of residence time of the main belt to the near-Earth region, the absolute value of those ages would change, but these ages would change proportionally to each other. Thus, if the ages are estimated under the main belt condition, the ages would be up to about 4 times larger: the spectral ages are 2 -- 32 Myr \citep{bonal2015} and $<$40 Myr \citep{koga2014}, and the CRE ages are 3 -- 32 Myr \citep{nagao2011} and 1.5 -- 6 Myr \citep{meier2014}.
In contrast, crater age and boulder age has opposite relationship to the residence time ratio of the main belt to the near-Earth region. As the residence time in the main belt is longer, both crater retention age and boulder distribution age would become shorter and spectral age and CRE age would become longer.

The CRE ages are the residence time for regolith particles in the top several meters of layer. These ages may reflect similar to what our crater retention age may present ($\sim$10 m). Moreover, the ``boulder exposure age'' is also an impact-induced age and is expected to have value similar to the crater retention age. The boulder exposure age can be estimated from the boulder size distribution on the surface of Itokawa. \citet{basilevsky2014} estimated as 5 -- 75 Myr based on the impact disruption probability of meter size surface boulders. These ages are compatible with our crater retention ages, which also reflects the resurfacing age information of $\sim$10 m of depth. This will also help us cross-examine the accuracy of our crater scaling law on coarse-grained targets. Such agreement supports that the crater retention age for Itokawa obtained in this study is not much lower than the true value.

Theoretical calculations have suggested that asteroids become smaller through both a catastrophic disruption of the parent body and subsequent impacts: collision cascade. Dynamical lifetimes are shorter for smaller asteroids. The crater resurfacing timescale for the upper $\sim$10 m layer, which is estimated from craters $>$ 100 m has important implications for collisional history of Itokawa. The fact that the crater retention age estimated based on a strength scaling by \citet{michel2009} overlaps with the Ar degassing age by \citet{park2015}, which is possibly due to a catastrophic break up of the parent body. The Ar degassing age may lead one speculate that the old crater retention age would reflect the catastrophic disruption of the parent body. However, our crater retention age based on the new scaling law for coarse-grained targets, which is more realistic conditions for the rubble pile asteroid Itokawa, may reflect a much more recent event, such as a subsequent disruption or global resurfacing, on Itokawa in the main belt long after catastrophic disruption of a rigid parent body. Such subsequent disruption and/or global resurfacing event after the initial breakup from the parent body is not surprising, considering its short collisional lifetime due to its small size \citep[e.g.,][]{obrien2005}.

Another remarkable characteristics seen among different ages shown in Figure \ref{itokawa-ages} is a large gap among the upper most layer ages. Long spectral ages $\sim 10^6$--$10^7$ Myr and vert short sample ages based on rim thickness and Ne implantation $\sim 10^2$--$10^4$ yr. It is important to understand such large discrepancy in surface ages. First, there are two spectral ages; one is the average spectral age from remote sensing data \citep{koga2014} and another is the spectral age of individual regolith grains obtained by Hayabusa \citep{bonal2015}. Their ages, $<$10 Myr and $\sim$2--8 Myr, are consistent each other.
 However, the spectral variation of ordinary chondritic material is estimated based on experiments by \citet{strazzulla2005}, which consider only Ar in the solar wind. A more resent study, \citet{loeffler2009}, showed that He ions could be more effective for space weathering because of large flux and make the space weathering of Olivine sample ($<$45 $\mu$m) saturated in $\sim$13,000 yr which is much faster than the time scale of space weathering by Ar ions $\sim$ 1.3 Myr. Because the spectra of both Itokawa particles and remote sensing data has not been saturated yet, the exposure time of upper most grains, i.e., microscopic exposure age, would be 10 times shorter when the effective He ion irradiation is taken into account. Thus, the spectral age of individual grains may be on the order of $10^3$ yrs. On the other hand, using the same method as \citet{bonal2015}, \citet{brunetto2006b} estimated the spectral age of Karin family $\sim$2 Myr. This age is consistent with the possible disruption age of the parent body of Karin family estimated based on the orbital distribution and the N-body simulation by \citet{nesvorny2002}. The ages estimated based on the heavy ion irradiation experiment \citep{strazzulla2005} would be apparent ages (i.e., macroscopic exposure ages) including partial rejuvenation of asteroid surface by granular convection or regolith disruption. Thus, the spectral ages of Itokawa $\sim$ 1--10 Myr \citep{koga2014,bonal2015} which are estimated based on the experimental result by \citet{strazzulla2005}, may also indicate the global apparent age including rejuvenation due to surface gardening. Thus, the effective thickness of the layer that the ``macroscopic ages'' represents would be much greater than grain diameters. Although it is difficult to accurately estimate this effective thickness, theoretical modeling and surface observations suggest on the order of cm to meters \citep{tancredi2012, yamada2016}.

Uppermost surface ages have also been estimated based on solar wind damaged rim thickness and Ne deposition of sample particles. The thickness of rim created by solar wind of olivine grains is estimated to saturate in $\sim 10^4$ yr and reaches to 100 nm thick \citep{keller2016}. However, the solar wind damaged rim thickness of Itoakwa samples is $\sim$ 60--70 nm. Thus, the exposure ages of samples are several thousand years \citep{berger2015}. Exposure times for different samples based on measurements of deposited Ne are 150, 410, and 550 years ignoring the erosion of grain surfaces and backscatter effect of Ne ions, they give lower limits. However, the actual irradiation ages were about several times of these ages \citep{nagao2011}; there is a very low possibility that the rim thickness ages are much longer than several thousand years.

Thus, these ages of uppermost layer may reflect the different timescales of the Itokawa surface layer, $10^6$--$10^7$ yr and $10^2$--$10^4$ yr. These two groups of ages may be a macroscopic age and a microscopic age, respectively.

More specifically, the microscopic age ($10^2$--$10^4$ yr) is the exposure age of upper most grains. This short exposure age indicates the high mobility of regolith of Itokawa, for example migration and convection of regolith \citep{miyamoto2007}. Moreover, this sample ages indicate that the timescale of space weathering for individual particles would be much shorter than it is previously thought \citep{strazzulla2005}.

In contrast, the macroscopic age ($10^6$--$10^7$ yr) is the apparent exposure age of a regolith layer, reflecting complex processes of regolith, such as regolith gardening, granular convection, and regolith fragmentation. Because our crater retention age is also a macroscopic age, it is more appropriate to compare directly with this age. The coincidence of the spectral age and our crater retention age suggests that a global resurfacing did not occur for $\sim$10 Myr. For a decisive conclusion, more analyses about the ``apparent spectral age'' are needed particularly with irradiation of realistic composition of ions for solar wind. 

Finally, those analyses and interpretations of the surface processes of small asteroids would be also useful for landing site selection for on-going asteroid sample-return missions, such as Hayabusa2 and OSIRIS-REx.
\red{When we assume similar surface conditions for Ryugu and Bennu, the new scaling law eq.\eqref{new-pi} could be applied.} 

\begin{figure}[htbp]
	\begin{center}
	\includegraphics[width=150mm]{/Users/eri/Dropbox/2016icarus/Figures/fig19_new.pdf}
	\caption{(a) Crater diameter on Itokawa predicted from the new scaling law with the target grain diameter $D_t=2$ m and(b) that normalized by the classic gravity scaling. Black dashed line is the classic $\pi$ gravity scaling \citep[e.g.][]{holsapple1993} for fine sand on Itokawa and gray dashed line is the classic $\pi$ strength scaling for soft rock \citep{holsapple1993}. Solid curves are new scalings proposed in this study; the complete new scaling eq.\eqref{new-pi} (red), new gravity scaling for Itokawa (blue) and new disruption strength scaling for Itokawa (magenta).  The curve for the new gravity scaling approaches fine-sand gravity scaling as projectile diameter increases. Note that the solid black line indicates the disruption limit of one grain ($\xi=1$). When projectile size is smaller than the limit, crater size might follow the classic strength scalings. A small crater on target grain may form. }
	\label{Itokawa-crater}
	\end{center}
\end{figure}

\begin{figure}[htbp]
	\centering
	\includegraphics[width=120mm]{/Users/eri/Dropbox/2016icarus/Figures/fig20.eps}
	\caption{Comparison among Itokawa surface ages estimated from the analyses of returned samples and the remote-sensing data\citep{obrien2005, michel2009, nagao2011, basilevsky2014, koga2014, meier2014, noguchi2014, berger2015, bonal2015, nishiizumi2015, park2015}. The solid lines indicate original literature values based on near-earth orbits. Dashed-lines indicate ages recalculated with a reduced Solar wind flux in the main belt; 1/4 that at the current near-Earth orbit. The age estimates with asterisks maybe ``macroscopic'' surface ages, which may reflect surface rejuvenation due to granular convection and regolith grain disruption. Thus, the effective thickness of the layers that these ages with asterisks represent may be much greater than grain diameter. See the main text for a detailed discussion. }
	\label{itokawa-ages}
	\centering
\end{figure}



%------------------------------------------------------------%
 \subsection{Implications for dynamical evolution of the asteroid main belt}
%------------------------------------------------------------%
The size frequency distribution (SFD) is an important factor that controls mass flux induced by the radial drift of asteroids through Yarkovsky effect.
Such drift may have played an important role in transport in volatiles, organic compounds, and water to the inner solar system.
The SFD of MBAs have evolved by innumerable collisions. Excavation and disruption due to mutual collisions grind asteroids to smaller sizes. A recent theoretical calculation showed this collisional cascade is controlled mainly by cratering (small impacts) rather than catastrophic disruption (large impacts) on collisions \citep{kobayashi2010}.
This suggests that the power-law index of SFD could be influenced greatly by mass loss of asteroids by cratering.
Thus, cratering efficiency on small asteroids plays an important role in controlling the size frequency distribution of smaller asteroids.
Furthermore, smaller asteroids become projectiles to larger asteroids and terrestrial planets and control the impact rate to them.

Our experiments suggest that large craters ($D_c>100$ m) on Itokawa are influenced by the disruption strength of surface grains but that the effect was small enough to form crater several times larger than on monolithic rock targets.
The material disruption strength starts to influence the crater size below $\sim 5$ m under such surface conditions. This can help the drastic depletion in small crater population $< 10$ m on Itokawa.
%Thus, the volume efficiency of craters on coarse-grained surface would be 0.125 to 1 times of the gravity scaling.
In other words, Itokawa may be protected well by the armoring effect for crater size $<10$ m.

However, Itokawa is not very well protected by armoring effect for larger size impacts.
This may actually have important implications for mass loss rate for such small rubble-pile bodies.
The results of this study predicts that impacts which create craters $>10$ m would generate ejecta mass as great as that predicted by the classic gravity scaling.
Then, small rubble-pile asteroids easily lose their masses due to small escape velocities.
For example, the escape velocity of Itokawa is less than 1 m/s; the impact ejecta faster than 1 m/s would not come back to the body.
Impact-excavated volume by small (a few meter) meteorites on rubble-piles may be larger than that on monolithic bodies by orders of magnitude.
This leads to the erosion rate for rubble piles by orders of magnitude higher than the monolithic body estimated by \citet{kobayashi2010}.
Thus, the SFD evolution of MBAs might be influenced greatly by internal structure of asteroids.
Effective impact-induced ``erosion'' by cratering suggests a high mass loss rate for small rubble-pile asteroids, which might contribute to eliminating small asteroids from the main belt and result in shallower distribution of small asteroids $0.15 <D_a<10$ km \citep{yoshida2007,gladman2009}.


 %%%%%%%%%%%%%%%%%%%%%%%%%%%%%%%%%%%%%%%%%%%%%%%%%%%%%%%%%%%%%%%%%%%%%%%%%%
 \section{Conclusions}\label{sec:conclusion}
 %%%%%%%%%%%%%%%%%%%%%%%%%%%%%%%%%%%%%%%%%%%%%%%%%%%%%%%%%%%%%%%%%%%%%%%%%%
In order to understand the cratering mechanism on boulder-rich surfaces and to evaluate the cratering efficiency, we conducted a series of impact experiments on coarse-grained targets with four kinds of materials: pumice, basalt, soda glass, and sintered-glass-beads.
We measured the diameter and topographic profile of craters and found both the size ratio $\psi\equiv D_p/D_t$ of impactor to the target grain and the energy ratio $\xi\equiv \frac{1}{2}m_pU^2/Q_D^*m_t$ of impact energy to the disruption energy of a target grain are important for understanding cratering efficiency reduction, so-called armoring effect.
The way armoring effect occur could be classified into three regimes as a function of the energy and size ratios of impactors to targets: (1) gravity regime: $\psi>1$ or $\xi\gg 1$, (2) armoring regime I (reduced size crater): $\psi<1$ and $\xi\gtrsim 1$, (3) armoring regime \II\,(no apparent inter-grain crater): $\psi<1$ and $\xi<1$.
We derived a new crater scaling law eq.\eqref{new-pi} for conditions (1) and (2).
\red{We found that crater diameters on coarse-grained targets can form 3 times smaller than that on sand targets at most when the asteroid surface is covered with 2 m blocks (Figure \ref{Itokawa-crater}).}

Impact cratering with armoring effect comprises of two stages: the disruption stage and the excavation stage.
The grain disruption stage completes very rapidly during the impact process.
The subsequent excavation stage occurs after the disruption, and its flow field is similar to that for simple crater formation when the impact energy is sufficiently high.
This observation suggests that the impactor/target size ratio and disruption strength affects only the fracture stage and that modification for the scaling parameters in widely used $\pi$ scaling \citep[e.g.,][]{holsapple1993} can make this scaling reproduce experimental data for coarse-grained targets over a wide range of impact velocities.
We also found that the modified $\pi$ parameters can be calculated based on momentum conservation during the first contact between from the impactor and the target grain.

The cratering efficiency (i.e., $\pi_V^*$) for the armored cratering is at least few times larger than the strength scaling for monolithic bodies whenever the impactor kinetic energy is higher than the disruption energy of the first-contact target grain (i.e., $\xi>1$). Note that an impactor at the average impact velocity in the main asteroid belt catastrophically disrupts a surface grain more than 20 times its own size due to hypervelocity impacts. Consequently, even if the surface grain size is as large as $\sim$meter, the armoring effect may not be effective for large craters made by meter-size impactors, because a meter-size boulder could be disrupted easily by a cm-size impactor. In such cases, crater size would be comparable to the size predicted by the classic gravity scaling.

The new scaling law for coarse-grained targets allows us to predict that five 100-m diameter crater-like depressions on Itokawa may have been made over 10 -- 33 Myr of time, which is still much younger than the previous estimated age (75 Myr -- 1 Gyr) based on strength scalings \citep{michel2009}.
%Because the obtained crater retention ages are significantly longer than the average orbital lifetime of Itokawa in a near-Earth orbit, the observed five circular depressions $> 100$ m are likely to have been formed in the main belt if they are of impact origin.
The possibility that forming this number of 100 m-class craters on in the near-Earth is very low ($<10$\%).
The CRE ages of from several to 10 Myr which would reflect the age for the top several m of thickness \citep{nagao2011, meier2014, nishiizumi2015} are consistent with or slightly shorter value than  our crater retention ages which reflects the age information of $\sim 10$ m deep.
These lines of evidence strongly suggest that Itokawa may have experienced a global resurfacing event in the main belt in its recent history, a few tens of million years ago, long after a catastrophic disruption of its rigid parent body, perhaps $\sim 1.3$ Ga as suggested by Ar-Ar age \citep{park2015}.
The large difference in age between these two events may suggest that Itokawa may have experienced at extensive resurfacing event(s) since a catastrophic disruption of its parent body.
Our results are generally in agreement with theoretical prediction on collisional evolution of small asteroids.
A simple calculation based on our new scaling law for coarse grained surface indicates that mass loss from small rubble-pile asteroids, such as Itokawa, Ryugu, and Bennu, may be much larger than that of monolith body and rather close to a strengthless body against medium to large impacts.
This may contribute to removing a small asteroid from the main belt through the Yarkovsky effect.

 %%%%%%%%%%%%%%%%%%%%%%%%%%%%%%%%%%%%%%%%%%%%%%%%%%%%%%%%%%%%%%%%%%%%%%%%%%
\section*{Acknowledgement}
We thank Prof. M. Arakawa, Prof. A. Nakamura from Kobe University and Prof. T. Kadono of University of Occupational and Environmental Health, Japan who provided insights and fruitful discussions. We also thank Dr. S. Hasegawa, Dr. A. Suzuki, and Dr. T. Hirai from the Space Plasma Laboratory in ISAS/JAXA who helped our experiments. A part of this research experiments were conducted with the Hypervelocity Impact Facility supported by JAXA inter-university research system.
The research was supported by the Japan Society for the Promotion of Science (JSPS) KAKENHI (Grant Number 26247092 and 15J06330) and Core-to-Core program ``International Network of Planetary Sciences''.

 %%%%%%%%%%%%%%%%%%%%%%%%%%%%%%%%%%%%%%%%%%%%%%%%%%%%%%%%%%%%%%%%%%%%%%%%%%
 \appendix
 %------------------------------------------------------------%
\section{Boulder distribution on Itokawa} \label{boulder-dist}
%------------------------------------------------------------%
When the new scaling law is derived for a given size of target grains. However, real rubble-pile asteroids are covered with a variety of grain sizes of grains.
In order to use our new scaling law for a targets with grains with different sizes one approach is to derive a characteristic size to represent the surface. In this section, we describe how we treat the size distribution and decide the characteristic size of boulders.

Rocks and boulders on airless bodies often exhibit fractal size distributions; the cumulative size distributions are given by power laws, such as
%Generally the cumulative distributions of boulder size on planetary surfaces satisfy fractal features and can be represented with exponential dependence as
\begin{equation}
	N_b(>D) \propto D^{\alpha}, \label{cdf}
\end{equation}
where $N_b(>D)$ is the number of boulders with a diameter greater than $D$.
The same boulder size distribution can be described in an incremental size distribution as $n_b(D){\rm d} D =- {\rm d} N_b$.

The power-law index $\alpha$ on Itokawa was first measured by \citet{saito2006} for the boulder size down to 5 m as $-2.8$ where the maximum size dimension of a each boulder was used as its diameter.
Subsequently, \citet{michikami2008} conducted a more detailed analysis and found that the power-index is $\alpha=-3.1\pm 0.1$.
Further analyses conducted more recently by \citet{mazrouei2014} and \citet{tancredi2015} revealed that power-law index may differ among different regions ranging from $-4.1$ to $-2.7$.

\citet{tancredi2015} have analyzed boulder size distributions for several regions with high-resolution images from 0.01 to 0.48 m/pix.
Due to the limitation of resolutions, small particles could not be counted.
%We use the data of boulder distributions with the size range of ten times larger than the resolution and assess whether the power law breaks or not.
\citet{tancredi2015} used a maximum-likelihood fitting method with goodness-of-fit tests based on the Kolmogorov-Smirnov statistic to determine the power-law index and the usable size range of fitting, while the previous studies fitted the exponents by eye.
They found that the cut-off sizes for the power laws for boulders are $\sim 2$ m for the rough terrain and $0.2 - 0.6$ m for the smooth terrain, which are significantly larger than the cut-off diameter due to artificial incompleteness bias from limitation in resolution.

Random packing simulations by \citet{tancredi2015} suggest that the surface size distribution might be a good representative for the interior size distribution (App. A).
The total volume $\theta_{\rm total}$ of boulders that follow the size distribution \eqref{cdf} is given by
\begin{align}
	\theta_\text{total}&=\int_{D_{\rm min}}^{D_{\rm max}} \frac{\pi}{6} D^3 n_b(D) {\rm d} D \notag \\
%	&=\int_{d_{\rm min}}^{d_{\rm max}} \frac{\pi \alpha C}{6} d^{2+\alpha} \d d \notag \\
	&=\begin{cases} -\dfrac{\pi \alpha C}{6(3+\alpha)} (D_{\rm max}^{3+\alpha}-D_{\rm min}^{3+\alpha})& (\alpha\neq -3) \medskip\\
	\dfrac{\pi C}{2}\ln\left(\dfrac{D_{\rm max}}{D_{\rm min}}\right) & (\alpha=-3) \end{cases} .
\end{align}
The characteristic size of boulders could be defined as a volume median diameter $\bar{D}$ which refers to the midpoint boulder size, where one half of the volume is in boulders with smaller than this size and the other half of the volume is in boulders larger than this size.
\begin{equation}
	\frac{\ln (D_{\rm max})-\ln (\bar{D})}{\ln (D_{\rm max})-\ln (D_{\rm min})}=0.5\qquad (\alpha=-3).
\end{equation}

Here, we assume that the power-index $\alpha =-3$ and that the cut-off sizes are $D_\text{min}=0.1$ m, $D_\text{max}=40$ m for the rough terrains on Itokawa.
The volume fraction of boulders larger than $D_p$ is
\begin{equation}
	P(>D_p) = \frac{\theta(>D_p)}{\theta_\text{total}}=\frac{\ln (D_{\rm max})-\ln (D_p)}{\ln (D_{\rm max})-\ln (D_{\rm min})}. \label{P}
\end{equation}
Figure \ref{graph-P} shows the volume fractions for different boulder minimum cut-off sizes.
The probability that an impactor hits a surface grain larger than $\sim2$ m is greater than 50\% for $D_{\rm min}=0.1$ m.
Moreover, when we assume the minimum cut-off size 2 m for the rough terrain, an impactor hit a surface grain larger than 9 m with the probability of 50\%.
Thus, it is not rare to achieve the condition $D_p<D_t$, because small impacts occur very often.
Under such a condition, the first-contact grain size would be larger than the impactor size and significant armoring effect may occur.
\begin{figure}[htbp]
	\begin{center}
	\includegraphics[width=7cm]{/Users/eri/Dropbox/2016icarus/Figures/fig22.eps}
	\caption{Volume occupation ratio of target grains larger than the impactor size $d$ for different minimum cut-off sizes: eq.\eqref{P}.}
	\label{graph-P}
	\end{center}
\end{figure}

%%%%%%%%%%%%%%%%%%%%%%%%%%%%%%%%%%%%%%%%%%%%%%%%%%%%%%%%%%%%%%%%%%%%%%%%%%%%%%%%%%%%%%%%%%%%%%%%%%%
\section{Classic scaling law of impact process}\label{sec:piscaling}
%%%%%%%%%%%%%%%%%%%%%%%%%%%%%%%%%%%%%%%%%%%%%%%%%%%%%%%%%%%%%%%%%%%%%%%%%%%%%%%%%%%%%%%%%%%%%%%%%%%
Extrapolation from crater size data in laboratory scales to planet scales with different gravities and material conditions, scaling laws are needed.
%\sout{Cratering is a complex processes involving the balance in mass, momentum, and energy of the material and also the equation of states.}
\citet{holsapple1987} proposes a point-source measure that determines the final outcome, ``coupling parameter'':
\begin{equation}
	C=aU^\mu\rho_p^\nu,
\end{equation}
assuming to measure observable in the far field from the impactor with radius $a$, velocity $U$ and mass density $\rho_p$.
Then, they developed this relation to more generalized formulations based on the Buckingham $\pi$ theorem of dimensional analysis \citep{buckingham1914}.
Starting from the crater volume $V_c$:
\begin{equation}
	V_c=f[aU^\mu\rho_p^\nu,\, \rho_t,\,Y,\, g], \label{pi-start-eq}
\end{equation}
where $\rho_t$ and $Y$ are the density and the strength of the bulk target and $g$ is the surface gravity, they derived set of for dimensionless, $\pi$ group parameters,
\begin{equation}
	\pi_V=K_1\left[\pi_2\pi_4^{-1/3}+\bar{\pi}_3^{(2+\mu)/2}\right]^{-3\mu/(2+\mu)}.\label{simple-pi}
 \end{equation}
\begin{align}
	\pi_V&=\frac{\rho_t V_c}{m_p}, \label{piV}\\
	\pi_2&=\frac{g}{U^2}\left(\frac{m_p}{\rho_p}\right)^{1/3}=3.22\left(\frac{ga}{U^2}\right),\\
	\pi_3&=\frac{\bar{Y}}{\rho_tU^2}, \label{pi3}\\
	\pi_4&=\frac{\rho_t}{\rho_p}.
\end{align}
The $\pi_V$ is called the cratering efficiency, where $m_p=(4/3)\pi a^3\rho_p$ is the projectile mass.
The $\pi_2$ is called the gravity-scaled size, and $\pi_3$ is called the non-dimensional strength indicating a measure of the importance of target strength $Y$ in cratering event.
 The parameters for geological materials are given in Table \ref{pi-params} from \citet{holsapple1993}.

\begin{table}[htbp]
	\centering
	\caption{Parameters for $\pi$ scaling \citep{holsapple1993}.}
	\label{pi-params}
	\small
	\begin{tabular}{lccc}\hline
	Material & $K_1$ & $\mu$ & $\bar{Y}$ [MPa]\\ \hline
	Dry soil & 0.24 & 0.41 & 0.18 \\
	Water & 2.3 & 0.55 & 0 \\
	Soft rock & 0.20 & 0.55 & 7.6\\
	Hard rock & 0.20 & 0.55 & 18\\ \hline
	\end{tabular}
	\centering
\end{table}

Using this scaling law, the craters on Itokawa and the Moon are estimated in Figure \ref{impactor-crater}.
Crater size can be estimated from the strength scaling for small impacts and the gravity scaling for large impacts.
The transition point from the strength scaling to the gravity scaling depends on the effective strength of the target material.
The ``effective" strength may include material tensile strength, friction, inter-locking resistance, and adhesion. Surface gravity even affects the ``effective'' strength through friction.
Comparison between observations and theoretical calculations suggest the cohesive strength of small asteroids is a few tens Pa \citep{sanchez2014, rozitis2014}.
If the effective strength of Itokawa was a few tens Pa, the transition crater diameter between strength-controlled to gravity-controlled regimes is smaller than 1 m.
Thus, crater size $> 1$ m can be predicted well with the gravity scaling if the target is fine regolith.
Note that crater size on a rock target predicts less than tenth the crater size on sand with small effective strength (1 kPa).
\begin{figure}[htbp]
	\centering
	\includegraphics[width=80mm]{Figures/fig21.eps}
	\caption{Crater sizes on Itokawa and Lunar estimated from the classic $\pi$ scaling eq. \eqref{simple-pi}. Gravity-dominated scaling (solid), strength-dominated scaling for sand with $\bar{Y}=1 {\rm MPa},\, 1{\rm kPa},\, 1{\rm Pa}$ in order from the top (dashed) and strength scaling for continuum target with $\bar{Y}=18$ MPa corresponding to the strength of hard rock(one-dot). Cratering process in small impacts are controlled by material strength and as an impact size increases, a transition occurs and a crater size follows the gravity scaling. The transition point from the strength scalings to the gravity scaling depends on effective strength and gravity.}
	\label{impactor-crater}
	\centering
\end{figure}

%%%%%%%%%%%%%%%%%%%%%%%%%%%%%%%%%%%%%%%%%%%%%%%%%
\section*{References}
\bibliography{alphabet}
%%%%%%%%%%%%%%%%%%%%%%%%%%%%%%%%%%%%%%%%%%%%%%%%%

\end{document}
